\documentclass[a4paper,12pt]{jsarticle}

% 余白
\usepackage[top=20truemm, bottom=25truemm, left=22truemm, right=22truemm]{geometry}
% 数式
\usepackage{amsmath}
\usepackage{mathtools}
\mathtoolsset{showonlyrefs,showmanualtags} 	 % 相互参照した式のみに番号を振る
% 画像
\usepackage[dvipdfmx]{graphicx}
\usepackage[subrefformat=parens]{subcaption}
\captionsetup{compatibility=false}
% ハイパーリンク
\usepackage[dvipdfmx]{hyperref}
\usepackage{pxjahyper}

% コマンド定義
\def\vec#1{\mbox{\boldmath $#1$}}
\newcommand{\dif}[2]{\frac{{\rm d} #1}{{\rm d} #2}}
\newcommand{\pdif}[2]{\frac{\partial #1}{\partial #2}}

\title{有効断面積からポテンシャルの推定}

\begin{document}
\maketitle

(18.2)より
\begin{equation}
	{\varphi}_0 = \int_{r_{\rm min}}^{\infty} \cfrac{\rho \cfrac{{\rm d}r}{r^2}}
	{\sqrt{1 - \cfrac{\rho^2}{r^2} - \cfrac{U}{E}}}
	\label{eq:phi}
\end{equation}
である。
ここで,この被積分関数は常に正であるから,
$\varphi_0$は$r_{\rm min}$が減少することで増加する。
$r_{\rm min}$は被積分関数の分母の根であるから,$\rho$が増加することで増加する。
つまり,$\varphi_0$は$\rho$の単調減少関数になっている。
$U$は斥力だから$0 < \varphi_0 < \pi/2$である。
したがって,$\chi = \pi - 2\varphi_0$であり,
$\chi$は$\rho$の単調増加関数になっている。

これらを踏まえると,原点に向かって飛んできた($\rho = 0$)粒子は$\chi = \pi$であり,
今衝突パラメータが$\rho^{\prime}$である粒子が飛ぶ方向を$\chi^{\prime}$とすると,
$0 < \rho < \rho^{\prime}$にある粒子は$\chi^{\prime} < \chi < \pi$の間に散乱する。
そこで公式
\begin{equation}
	\int_{\chi^{\prime}}^{\pi} \dif{\sigma}{\chi} {\rm d}\chi
	= \pi {\rho^{\prime}}^2
\end{equation}
が成り立つ。

式(\eqref{eq:phi})において,
\begin{equation}
	s = \frac{1}{\rho}, \qquad x = \frac{1}{\rho},
	\qquad w = \sqrt{1 - \frac{U}{E}}
\end{equation}
を用いると,
\begin{equation}
	\rho \int_0^{s_0} \frac{{\rm d}s}
	{\sqrt{xw^2 - s^2}}
\end{equation}
となる。
ここで,$s_0$は分母の根である。

\end{document}