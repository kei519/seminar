\documentclass[a4paper]{jsarticle}

% 余白
\usepackage[top=20truemm, bottom=25truemm, left=22truemm, right=22truemm]{geometry}
% 数式
\usepackage{amsmath, amssymb}
\usepackage{ascmac}
\usepackage{mathtools}
\mathtoolsset{showonlyrefs,showmanualtags} 	 % 相互参照した式のみに番号を振る
% 画像
\usepackage[dvipdfmx]{graphicx}
\usepackage[subrefformat=parens]{subcaption}
\captionsetup{compatibility=false}
% ハイパーリンク
\usepackage[dvipdfmx]{hyperref}
\usepackage{pxjahyper}

% コマンド定義
\def\vec#1{\mbox{\boldmath $#1$}}
\newcommand{\dif}[2]{\frac{{\rm d} #1}{{\rm d} #2}}
\newcommand{\pdif}[2]{\frac{\partial #1}{\partial #2}}
\newcommand{\ddif}{{\rm d}}

\begin{document}

\begin{align}
	\dif{}{t} \left( a_{ij}(Q) \dot{\xi_j} \right) = f_i(Q, t)
\end{align}
で、$Q$に依存する量はほぼ変化しないから定数とみなして、
\begin{align}
	 &a_{ij} \ddot{\xi_i} = f_i (Q, t) =
	 f_i^{(1)} (Q) \cos \omega t + f_i^{(2)} (Q) \sin \omega t \\
	 &\ddot{\xi_i} = a_{ij}^{-1}
	 \left( f_j^{(1)} \cos \omega t + f_j^{(2)} \sin \omega t \right)
\end{align}
を得るから、
\begin{align}
	 \xi_i = -\frac{a_{ij}}{\omega^2}
	 \left( f_j^{(1)} \cos \omega t + f_j^{(2)} \sin \omega t \right)
	 \label{eq:xi}
\end{align}
となる。

また、
\begin{align}
	 \overline{f_i f_j} = \frac{1}{2}
	 \left( f_i^{(1)} f_j^{(1)} + f_i^{(2)} f_j^{(2)} \right)
	 \label{eq:f}
\end{align}
であるが、式\refeq{eq:xi}より
\begin{align}
	 \dot{\xi_i} = \frac{a_{ij}^{-1}}{\omega}
	 \left( f_j^{(1)} \sin \omega t - f_j^{(2)} \cos \omega t \right)
\end{align}
であることより、
\begin{align}
	\overline{\dot{\xi_i}\dot{\xi_j}} &= \frac{a_{ik}^{-1} a_{jl}^{-1}}{\omega^2} 
	\overline{\left( f_k^{(1)} f_l^{(1)} \sin^2 \omega t
	- f_k^{(1)} f_l^{(2)} \sin \omega t \cos \omega t
	- f_k^{(2)} f_l^{(1)} \cos \omega t \sin \omega t
	+ f_k^{(2)} f_l^{(2)} \cos^2 \omega t \right)} \\
	&= \frac{a_{ij}^{-1} a_{jl}^{-1}}{2\omega^2}
	\left( f_k^{(1)} f_l^{(1)} + f_k^{(2)} f_l^{(2)} \right)
\end{align}
となるから、式\refeq{eq:f}より
\begin{align}
	\overline{f_i f_j} = \omega^2 a_{ik} a_{jl}
	\overline{\dot{\xi_k} \dot{\xi_l}}
\end{align}
と書ける。
したがって、
\begin{align}
	U_{\mbox{有効}} &= U + \frac{a_{ij}^{-1}}{2\omega^2} \overline{f_i f_j} \\
	&= U + \frac{1}{2} a_{ij}^{-1} a_{ik} a_{jl}
	\overline{\dot{\xi_k} \dot{\xi_l}} \\
	&= U + \frac{a_{kl}}{2} \overline{\dot{\xi_k} \dot{\xi_l}}
\end{align}
となることが分かる。

\end{document}