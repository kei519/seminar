\documentclass[a4paper]{jsarticle}

% 余白
\usepackage[top=20truemm, bottom=25truemm, left=22truemm, right=22truemm]{geometry}
% 数式
\usepackage{amsmath, amssymb}
\usepackage{ascmac}
\usepackage{mathtools}
\mathtoolsset{showonlyrefs,showmanualtags} 	 % 相互参照した式のみに番号を振る
% 画像
\usepackage[dvipdfmx]{graphicx}
\usepackage[subrefformat=parens]{subcaption}
\captionsetup{compatibility=false}
% ハイパーリンク
\usepackage[dvipdfmx]{hyperref}
\usepackage{pxjahyper}

% コマンド定義
\def\vec#1{\mbox{\boldmath $#1$}}
\newcommand{\dif}[2]{\frac{{\rm d} #1}{{\rm d} #2}}
\newcommand{\pdif}[2]{\frac{\partial #1}{\partial #2}}

\title{\S\ 30.\ 急激に振動する場における運動}

\begin{document}
\maketitle

ここではある場$U$の中で運動している物体に,速く振動するような力が働く場合を考える.

まずは簡単のために,1次元での運動を考察する.
不変な場$U$の中で,周期的に運動している物体があるとする.
その周期の大きさを$T$としたときに,$\omega \ll 1/T$を満たす速い振動数で振動する力
\begin{equation}
	f(x, t) = f_1(x) \cos \omega t + f_2(x) \sin \omega t
\end{equation}
の作用も受ける状況を考える.
この場合の変位は,滑らか(後で説明)な変位$X$に振動的な変位$\xi$が加わった,つまり
\begin{equation}
	x(t) = X(t) + \xi(t) \label{eq:x}
\end{equation}
と書けるような運動になると考えられる.
ここでは$f$は小さいとはしないが,
$\xi \ll X$と仮定して(そのような状況のみを考えて)考察を進める.

このとき,$f$の周期$2\pi/\omega$で物理量$Q(t)$の時間平均を取ったものを,
\begin{equation}
	\overline{Q} = \int_t^{t + \frac{2\pi}{\omega}} {\rm d}t^{\prime} Q(t^{\prime})
\end{equation}
と表すことにすると,
$f$はそれ自身周期$2\pi/\omega$の三角関数だから,$\overline{f} = 0$となる.
また場$U$下での変位からの差$xi$は,
周期的な力$f$によって引き起こされていると考えられるから,
これも$\overline{\xi}=0$である.
また$X$の周期はだいたい$T \gg 1/\omega$であるから,$2\pi/\omega$ではほぼ変化せず,
$\overline{X} \simeq X$である.
つまり$\overline{x} = X$となり,
これは振動よりも遅い時間スケールで見れば,ほとんど$X$の運動しか見えないことになる.
これが$X$が滑らかといった意味である.

このとき,物体の運動方程式は
\begin{equation}
	m\ddot{x} = -\dif{U}{x} + f \label{eq:EOM}
\end{equation}
と書ける.
ここで式\eqref{eq:x}を用いると,$\xi$の1次の項まで取って,
\begin{align}
	\pdif{U}{x}(X+\xi) &\simeq \dif{U}{x}(X) + \xi \dif{}{x}\dif{U}{x}(X) \\
	&\simeq \dif{U}{X} + \xi \frac{{\rm d}^2 U}{{\rm d}x^2} \qquad \\
	f(X + \xi, t) &\simeq f(X, t) + \xi\pdif{f}{x} \\
	&\simeq f(X, t) + \xi\pdif{f}{X}
\end{align}
となる(ここで最後の等式において$x = X + \xi \simeq X$を用いた).
すると式\eqref{eq:EOM}は
\begin{equation}
	m\ddot{X} + m\ddot{\xi} =
	-\dif{U}{X} - \xi \frac{{\rm d}^2 U}{{\rm d}X^2}
	+ f(X, t) + \xi \pdif{f}{X}
	\label{eq:EOM2}
\end{equation}
と書ける.

$\xi$すなわち微小振動項は,振動している力によって起こされる変位であると考えられるから,
\eqref{eq:EOM2}からその項だけを取り出して,
\begin{equation}
	m\ddot{\xi} = f(X, t)
\end{equation}
が成り立つ.
ここで$\xi$の他の項については,微小な$\xi$が掛かっているため無視できる.
$\ddot{\xi}$は速く変化する($\omega^2$に比例する)ため,無視できない.
これは$X$の変化が$\xi$に比べて遅いから,簡単に解けて
\begin{equation}
	\xi = -\frac{f}{m\omega^2}
\end{equation}
となる.

ここまでの関係を用いて式\eqref{eq:EOM2}を整理すると,
\begin{equation}
	m\ddot{X} = -\dif{U}{X} - \xi \frac{{\rm d}^2U}{{\rm d}X^2} + \xi \pdif{f}{X}
\end{equation}
となる.
この式の時間平均を取ると,右辺の第2項は$X$が時間でほぼ変わらないことを考慮すると,
$\xi$に比例する項であるから,0になる.
したがって,
\begin{equation}
	m\ddot{X} = -\dif{U}{X} + \overline{\xi \pdif{f}{X}}
	= -\dif{U}{X} - \frac{1}{m\omega^2}\overline{f\pdif{f}{X}}
\end{equation}
と書き直せる.
これは有効ポテンシャル
\begin{equation}
	U_{\mbox{有効}} = U + \frac{1}{2m\omega^2} \overline{f^2}
	= U + \frac{1}{4m\omega^2}(f_1^2 + f_2^2)
	\label{eq:Ueff}
\end{equation}
を用いると,
\begin{equation}
	m\ddot{X} = -\dif{U_{\mbox{有効}}}{X}
\end{equation}
と書ける.
式\eqref{eq:Ueff}において,
\begin{align}
	&\int_{t}^{t+T} {\rm d}t^{\prime} \cos^2 \frac{2\pi t^{\prime}}{T}
	= \int_{t}^{t+T} {\rm d}t^{\prime} \sin^2 \frac{2\pi t^{\prime}}{T}
	= \frac{T}{2} \\
	&\int_{t}^{t+T} {\rm d}t^{\prime} \sin \frac{2\pi t}{T} \cos \frac{2\pi t^{\prime}}{T}
	= 0
\end{align}
と,$x \simeq X$が時間的に変動しないことを用いた.

同様にして,
\begin{equation}
	\overline{\dot{\xi}^2} = \frac{f_1^2 + f_2^2}{2m^2 \omega^2}
\end{equation}
と書き直せるから,
これを式\eqref{eq:Ueff}に用いると,
\begin{equation}
	U_{\mbox{有効}} = U + \frac{m}{2} \overline{\dot{\xi}^2}
\end{equation}
とも表される.
これは結局,まるで変動する場が存在して,
ポテンシャルがその振幅の2乗に比例するかのような運動が行われることになる.

質量$m$が場所$x$に依存する場合(デカルト座標を選ばなかった場合など)は,
教科書には「計算がいくらか長くなる」と書かれているが,
$\xi$を求める段階では$f$の$x$依存性のように,時間間隔が短いため定数と見なせて,
それ以後では時間平均を取るため,
ほぼ変更を受けずに有効ポテンシャルが得られると考えられる.

一般座標$q_i$で書かれていた場合,振動する力がないときのラグランジアン$L$は
\begin{equation}
	L = \frac{1}{2}a_{ij} \dot{q_i} \dot{q_j} - U(q)
\end{equation}
であるから,そこに座標$q_i$に力
\begin{equation}
	f_i = f_i^{(1)} \cos \omega t + f_i^{(2)} \sin \omega t
\end{equation}
が掛かっている状況を考えると,運動方程式は
\begin{equation}
	\dif{}{t} \left[ a_{ij}(Q) \left( \dot{Q_j} + \dot{\xi_j} \right) \right]
	= -\pdif{U}{Q_i} - \xi_j \frac{\partial^2 U}{\partial Q_j \partial Q_i}
	+ f_i (Q, t) + \xi_j \pdif{f_i}{Q_j}
\end{equation}
となる.
ここで1次元の場合と同じように$Q$,$\xi$を定義し,微小量$\xi$の2次以降を無視した.
また,同じように$\xi$についての方程式を抜き出すと,
\begin{equation}
	\dif{}{t} \left( a_{ij}(Q) \dot{\xi_j} \right) = f_i(Q, t)
\end{equation}
を得るが,ここでも$Q$にのみ依存する項は,$Q$が定数であるとみなしてよく,結局
\begin{align}
	\ddot{\xi_j} = a^{-1}_{ji} f_i(Q, t) \\
	\xi_i = -\frac{a^{-1}_{ij} f_j}{\omega^2} \label{eq:xi}
\end{align}
を得る.ここで$a^{-1}_{ij}$は$a_{ji}$の逆行列である.
すると,時間平均を取って$Q$についての方程式
\begin{equation}
	\dif{}{t} \left( a_{ij}(Q) \dot{Q_j} \right) =
	-\pdif{U}{Q_i} - \frac{a^{-1}_{jk}(Q)}{\omega^2}
	\overline{ f_k \pdif{f_i}{Q_j}}
\end{equation}
が同様にして得られる.
この$a^{-1}$は一般に$Q$依存性のみがあるから,
時間平均を取っても元の$a^{-1}(Q)$に一致する.

これも同様に有効ポテンシャル
\begin{equation}
	U_{\mbox{有効}} = U + \frac{a^{-1}_{ij}}{2\omega^2} \overline{f_i f_j}
\end{equation}
を導入すればよい.
また,式\eqref{eq:xi}を用いると
\begin{align}
	&f_i = a_{ij} \omega^2 \xi_j \\
	&\overline{f_i f_j} = \frac{1}{2}
	\left( f_i^{(1)} f_j^{(1)} + f_i^{(2)} f_j^{(2)} \right)
	= \omega^2 a_{ik} a_{jl} \overline{\dot{\xi_k} \dot{\xi_l}} \\
	&U_{\mbox{有効}} = U +
	\frac{1}{2} a^{-1}_{ij} a_{ik} a_{jl} \overline{\dot{\xi_k} \dot{\xi_l}}
	= U + \frac{a_{ij}}{2} \overline{\dot{\xi_i} \dot{\xi_j}}
\end{align}
と書ける.

\end{document}