\documentclass[a4paper]{jsarticle}

% 余白
\usepackage[top=20truemm, bottom=25truemm, left=22truemm, right=22truemm]{geometry}
% 数式
\usepackage{amsmath, amssymb}
\usepackage{ascmac}
\usepackage{mathtools}
\mathtoolsset{showonlyrefs,showmanualtags} 	 % 相互参照した式のみに番号を振る
% 画像
\usepackage[dvipdfmx]{graphicx}
\usepackage[subrefformat=parens]{subcaption}
\captionsetup{compatibility=false}
% ハイパーリンク
\usepackage[dvipdfmx]{hyperref}
\usepackage{pxjahyper}

% コマンド定義
\def\vec#1{\mbox{\boldmath $#1$}}
\newcommand{\dif}[2]{\frac{{\rm d} #1}{{\rm d} #2}}
\newcommand{\pdif}[2]{\frac{\partial #1}{\partial #2}}
\newcommand{\ddif}{{\rm d}}

\title{\S 36.\ オイラーの運動方程式}

\begin{document}
\maketitle

\S 34で与えられた運動方程式は、静止座標系で見たものである。
それを運動系のものに書き直すことを考えよう。
そうすれば、慣性主軸を座標軸として選ぶことで、角運動量と角速度の成分が簡単に書けることが
使える。

まずはじめに、ベクトル$\vec{A}$が回転系において変化しない場合を考察する。
そのとき、静止系で見れば、角速度$\vec{\Omega}$で$\vec{A}$が回転していることになる。
したがって、静止系での$\vec{A}$の時間微分は
\begin{align}
	\dif{\vec{A}}{t} = {\vec{\Omega} \vec{A}}
\end{align}
と書けることになる。
したがって、一般の(回転系においても$\vec{A}$が運動している)場合、
回転系での$\vec{A}$の時間微分を$\ddif^{\prime} \vec{A} / \ddif t$と書けば、
\begin{align}
	\dif{\vec{A}}{t} = \dif{^{\prime}\vec{A}}{t} + [\vec{\Omega} \vec{A}]
\end{align}
を満たす。

ここまでの考察を踏まえると、式(34.1)、(34.3)を
\begin{align}
	\dif{\vec{P}}{t} &= \dif{^{\prime}\vec{P}}{t} + [\vec{\Omega} \vec{P}]
	= \vec{F} \\
	\dif{\vec{M}}{t} &= \dif{^{\prime}\vec{M}}{t} + [\vec{\Omega} \vec{M}]
	= \vec{K}
\end{align}
と書き直せることが分かる。
これらを回転系の座標軸$(x_1, x_2, x_3)$の各成分ごとに書き直す。
\begin{align}
	[\vec{A} \vec{B}]_i = \epsilon_{ijl} A_j B_k
\end{align}
であることを用いれば、運動量に関する運動方程式は、
$\vec{P} = \mu \vec{V}$の関係式を使うことで、
\begin{align}
\begin{split}
	\mu \left( \dif{V_1}{t} + \Omega_2 V_3 - \Omega_3 V_2 \right) &= F_1 \\
	\mu \left( \dif{V_2}{t} + \Omega_3 V_1 - \Omega_1 V_3 \right) &= F_2 \\
	\mu \left( \dif{V_3}{t} + \Omega_1 V_2 - \Omega_2 V_1 \right) &= F_3
\end{split} \tag{36.4}
\end{align}
となる。
また角運動量似関しての運動方程式は、$M_i = I_i \Omega_i$を用いれば
\begin{align}
\begin{split}
	I_1 \dif{\Omega_1}{t} + (I_3 - I_2) \Omega_2 \Omega_3 &= K_1 \\
	I_2 \dif{\Omega_2}{t} + (I_1 - I_3) \Omega_3 \Omega_1 &= K_2 \\
	I_3 \dif{\Omega_3}{t} + (I_2 - I_1) \Omega_1 \Omega_2 &= K_3
\end{split} \tag{36.5}
\end{align}
となる。
ここで(自分が一瞬わからなかったので)注意しておくことは、
各$\vec{V}$、$\vec{M}$は、各成分の方向は時間によって変化するが、静止系での量である。

応用例として(また)自由な対称こまを考える。
まず対称軸を$x_3$に取ると、$I_1 = I_2$となるから、(36.5)式の3つ目の式から、
$\Omega_3 = {\rm const}$で有ることが分かる。
それを1,2の式に用いると、
\begin{align}
	\omega = \frac{I_3 - I_1}{I_1} \Omega_3
	\tag{36.5}
\end{align}
を用いることで、
\begin{align}
	\dif{\Omega_1}{t} = -\omega \Omega_2 \\
	\dif{\Omega_2}{t} = \omega \Omega_1
\end{align}
を得る。
これはよく知られた方法で解けて、第2式に$i$を掛けて両辺を足し合わせることで、
\begin{align}
	\dif{}{t} (\Omega_1 + \Omega_2) = -\omega \Omega_2 + i \omega \Omega_1
	= i \omega (\Omega_1 + i \Omega_2)
\end{align}
となるから、適当な初期条件(もしくは慣性主軸の選び方)の元、この解は
\begin{align}
	\Omega_1 + i \Omega_2 = A e^{i \omega t}
\end{align}
となる。
$\Omega_1$、$\Omega_2$は共に実数であるから
\begin{align}
	\Omega_1 = A \cos\omega t \quad
	\Omega_2 = A \sin\omega t
	\tag{36.7}
\end{align}
を得る。
これはつまり、$\vec{\Omega}$が$x_3$軸(対称軸)に垂直な平面内において、
大きさ一定で($\Omega_3 = {\rm const}$)角速度$\omega$で回転することを意味する。
このとき、$M_1 = I_1 \Omega_1$、$M_2 = I_1 \Omega_2$だから、
$\vec{M}$も$\vec{\Omega}$と同様に、$x_3$軸周りを回転する。

これを\S 33や\S 35で見たのと同様に、$\vec{M}$の向きを$Z$軸方向とすると、
\S 35で見たように、$Z$軸の方位角はオイラー角を用いると$\pi/2 - \psi$だから、
$x_3$軸に垂直な平面($x_1x_2$平面)に射影したときの角速度は$-\dot{\psi}$となる。
ここで(35.4)式を用いると、
\begin{align}
	\dot{\psi} = \frac{M \cos\theta}{I_3} - \dot{\phi} \cos\theta
	= M \cos\theta \left( \frac{1}{I_3} - \frac{1}{I_1} \right)
	= I_3 M \cos\theta \frac{I_1 - I_3}{I_1}
	= \Omega_3 \frac{I_1 - I_3}{I_1}
\end{align}
を得るが、これは
\begin{align}
	-\dot{\psi} = \Omega_3 \frac{I_3 - I_1}{I_1}
\end{align}
となり、(36.6)と一致する。

\end{document}