\documentclass[a4paper,12pt]{jsarticle}

% 余白
\usepackage[top=20truemm, bottom=25truemm, left=22truemm, right=22truemm]{geometry}
% 数式
\usepackage{amsmath}
\usepackage{mathtools}
\mathtoolsset{showonlyrefs,showmanualtags} 	 % 相互参照した式のみに番号を振る
% 画像
\usepackage[dvipdfmx]{graphicx}
\usepackage[subrefformat=parens]{subcaption}
\captionsetup{compatibility=false}
% ハイパーリンク
\usepackage[dvipdfmx]{hyperref}
\usepackage{pxjahyper}

% コマンド定義
\def\vec#1{\mbox{\boldmath $#1$}}
\newcommand{\dif}[2]{\frac{{\rm d} #1}{{\rm d} #2}}
\newcommand{\pdif}[2]{\frac{\partial #1}{\partial #2}}

\title{分布}

\begin{document}
\maketitle

$a < b$を用いて$x \in [a, b]$に何かが分布しているとき,
$x \in [p, q](a \le p \le q \le b)$に分布している割合が
\begin{equation}
	\int_p^q f(x) {\rm d}x
\end{equation}
となるとき,$f(x)$を分布(関数)と呼ぶ.
定義より,分布は以下の性質を満たす.
\begin{align}
	&\int_a^b f(x) {\rm d}x = 1 \\
	&f(x) \ge 0 \quad (x \in [a, b])
\end{align}
これを物理で用いるときは,$x \in [x, x + {\rm d}x]$に分布している割合を分布と呼び,
上の定義よりそれは,
\begin{equation}
	\int_x^{x + {\rm d}x} f(x^{\prime}) {\rm d} x^{\prime}
	= f(x) {\rm d}x
\end{equation}
となる.
以下これを用いる.

ここで,連続全単射(競技単調増加もしくは競技単調減少)である関数$y = y(x)$と,
その逆関数$x = x(y)$を用いて変数変換することを考えると,
\begin{equation}
	\int_p^q f(x) {\rm d}x
	= \int_{y(p)}^{y(q)} f(x(y)) \dif{x(y)}{y} {\rm d}y
\end{equation}
となる.
このとき条件より,${\rm d}x/{\rm d}y$は常に正もしくは負である.

正のとき,$y(x)$は狭義単調増加関数であるから,$y(p) < y(q)$となる.
したがって,$x \in [p, q]$に分布している割合,
すなわち$y \in [y(p), y(q)]$に分布している割合が
\begin{equation}
	\int_{y(p)}^{y(q)} f(x(y)) \dif{x}{y} {\rm d}y
\end{equation}
で与えられるから,
\begin{equation}
	f(x(y)) \dif{x}{y} {\rm d}y
\end{equation}
を$y$を用いたときの新たな分布とすればよい.

負のとき,$y(x)$は狭義単調減少関数であるから,$y(p) > y(q)$となる.
したがって,$x \in [p, q]$に分布している割合を$y$を用いて考えると,
$y \in [y(q), y(p)]$に分布している割合を考えることになる.
それは
\begin{equation}
	\int_{y(p)}^{y(q)} f(x(y)) \dif{x(y)}{y} {\rm d}y
	= \int_{y(q)}^{y(p)} f(x(y)) \left[ - \dif{x}{y} \right] {\rm d}y
\end{equation}
で与えられる.
これは,
\begin{equation}
	f(x(y)) \left[ - \dif{x}{y} \right] {\rm d}y
\end{equation}
を$y$を用いたときの新たな分布と考えることと同じである.

以上より,${\rm d}x/{\rm d}y$が正負の場合をまとめると,
\begin{equation}
	f(x(y)) \left| \dif{x}{y} \right| {\rm d}y
\end{equation}
を新たな分布とすることになる.

$y(x)$が単調でない場合は,極大極小を取る点で分割して,
それぞれの逆関数に$x = x_i (y)$と番号をつけて変数変換すれば上と同様に考えられる.

$y(x)$が多価の場合は,それぞれを$y_1,\ y_2,\ \cdots,\ y_n$のように番号をつけ,
それぞれについて上と同様に考えれば良い.

これらをすべてまとめると,結局$x \in [p, q]$の分布を変数変換して求めるときは,
${y(x) | x \in [p, q]}$の範囲で積分を実行すればよく,このときの分布は
\begin{equation}
	\sum f(x(y)) \left| \dif{x(y)}{y} \right| {\rm d}y
\end{equation}
とすればよい(ここでの和は全ての$x, y$についてとる).

\end{document}