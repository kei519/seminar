\documentclass[12pt]{article}

\usepackage[top=20truemm, bottom=25truemm, left=22truemm, right=22truemm]{geometry}
\usepackage{amsmath}


\begin{document}
	運動エネルギー$T(q, \dot{q})$が
	\[
	T(q, \dot{q}) = \sum_{i,j} a_{ij}(q) \dot{q_i} \dot{q_j}
	\]
	で表される時,
	\[
	\frac{\partial T}{\partial \dot{q_k}}
	= \sum_{i,j} a_{ij}(q) \left( \delta_{i,k} \dot{q_j} + \dot{q_i} \delta_{j,k} \right)
	=\sum_j 2a_{kj}(q) \dot{q_j}
	\]
	となる.
	ここで,$a_{ij}(q)$は対称となるように取る.
	また,
	\[
	\frac{\partial \dot{q_i}}{\partial \dot{q_j}} = \delta_{i,j}
	\]
	であることを用いた.
	
	したがって,
	\[
	\sum_{k} \dot{q_k} \frac{\partial T}{\partial \dot{q_k}}
	=\sum_{k,j} 2a_{kj}(q) \dot{q_k} \dot{q_j}
	=2T(q, \dot{q})
	\]
	である.
\end{document}
