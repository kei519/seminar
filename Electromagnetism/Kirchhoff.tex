\documentclass[a4paper, 10pt]{jsarticle}

% 余白
\usepackage[top=20truemm, bottom=25truemm, left=22truemm, right=22truemm, driver=dvipdfm, truedimen, margin=2cm]{geometry}
% 数式
\usepackage{amsmath, amssymb, amsthm}
\theoremstyle{definition}
\usepackage{ascmac}
\usepackage{mathtools}
\mathtoolsset{showonlyrefs,showmanualtags} 	 % 相互参照した式のみに番号を振る
% 画像
\usepackage[dvipdfmx]{graphicx}
\usepackage[subrefformat=parens]{subcaption}
\captionsetup{compatibility=false}
% ハイパーリンク
\usepackage[dvipdfmx, bookmarksnumbered]{hyperref}
\usepackage{pxjahyper}
\hypersetup{colorlinks=true, linkcolor=black, citecolor=black, urlcolor=black}

% コマンド定義
\def\vec#1{\mbox{\boldmath $#1$}}
\newcommand{\dif}[2]{\frac{{\rm d} #1}{{\rm d} #2}}
\newcommand{\pdif}[2]{\frac{\partial #1}{\partial #2}}
\newcommand{\ddif}{{\rm d}}
\DeclareMathOperator{\Div}{div}
\DeclareMathOperator{\Grad}{grad}
\DeclareMathOperator{\Rot}{rot}
\allowdisplaybreaks[4] 	 % 数式等のページ分割をさせる

\title{Kirchhoffの積分表示について}
\author{}

\begin{document}
\maketitle

\section{導出と体積積分項についての考察}
Greenの定理
\begin{align}
	\int_V \ddif^3 x' \left( \phi(\vec{x}', t') \Delta' \psi(\vec{x}', t')
	- \psi(\vec{x}', t') \Delta' \phi(\vec{x}', t') \right)
	= \int_{\partial V} \ddif S'
	\left( \phi(\vec{x}', t') \pdif{\psi(\vec{x}', t')}{n'}
	- \psi(\vec{x}', t') \pdif{\phi(\vec{x}', t')}{n'} \right)
\end{align}
から
\begin{align}
	&\int_{t_0}^{t_1} \ddif t' \int_V \ddif^3 x'
	\Bigl( D_{\rm ret}(\vec{x} - \vec{x}', t-t') \Delta' \psi(\vec{x}', t')
	- \psi(\vec{x}', t') \Delta' D_{\rm ret}(\vec{x} - \vec{x}', t-t') \Bigr) \\
	= &\int_{t_0}^{t_1} \ddif t' \int_{\partial V} \ddif S'
	\left( D_{\rm ret}(\vec{x} - \vec{x}', t-t') \pdif{\psi(\vec{x}', t')}{n'}
	- \psi(\vec{x}', t') \pdif{}{n'} D_{\rm ret}(\vec{x} - \vec{x}', t-t') \right)
\end{align}
となる。
ここで$\psi(\vec{x}, t)$は波動方程式
\begin{align}
	\left( \Delta - \frac{1}{c^2} \pdif{^2}{t^2} \right) \psi(\vec{x}, t) = 0
\end{align}
を満たす。
では両辺を評価していく。
まず左辺から考えると
\begin{align}
	(\mbox{左辺}) &=
	\frac{1}{c^2} \int_{t_0}^{t_1} \ddif t' \int_V \ddif^3 x'\left[
		D_{\rm ret}(\vec{x} - \vec{x}', t-t') \pdif{^2}{t'^2}
		\psi(\vec{x}', t')
		- \psi(\vec{x}', t')
		\left( \pdif{^2}{t'^2} D_{\rm ret}(\vec{x} - \vec{x}', t-t')
		- \delta(t-t') \delta(\vec{x} - \vec{x}') \right)
	\right]
\end{align}
となるから、$t_0 < t < t_1$かつ$\vec{x} \in V$であれば
\begin{align}
	(\mbox{左辺}) &= \frac{1}{c^2} \psi(\vec{x}, t)
	+ \frac{1}{c^2} \int_V \ddif^3 x' \int_{t_0}^{t_1} \ddif t' \pdif{}{t'}
	\left[ D_{\rm ret}(\vec{x} - \vec{x}', t-t') \pdif{}{t'} \psi(\vec{x}', t')
		- \psi(\vec{x}', t') \pdif{}{t'} D_{\rm ret}(\vec{x} - \vec{x}', t-t')
	\right]
\end{align}
となる。
ここで第2項には
\begin{align}
	\left[ D_{\rm ret}(\vec{x} - \vec{x}', t-t') \pdif{}{t'} \psi(\vec{x}', t')
		- \psi(\vec{x}', t') \pdif{}{t'} D_{\rm ret}(\vec{x} - \vec{x}', t-t')
	\right]_{t' = t_0}^{t_1}
\end{align}
という項が現れるが、
\begin{align}
	D_{\rm ret}(\vec{x} - \vec{x}', t-t')
	= \frac{1}{4\pi c} \frac{1}{|\vec{x} - \vec{x}'|}
	\delta \left( |\vec{x} - \vec{x}'| - c(t-t') \right)
\end{align}
であることと$t_0 < t < t_1$であることを思い出せば
\begin{align}
	(\mbox{左辺}) = \frac{1}{c^2} \psi(\vec{x}, t)
	- \frac{1}{c^2} \int_V \ddif^3 x'
	\left[ D_{\rm ret}(\vec{x} - \vec{x}', t-t_0) \pdif{}{t'} \psi(\vec{x}', t_0)
		- \psi(\vec{x}', t_0) \pdif{}{t_0} D_{\rm ret}(\vec{x} - \vec{x}', t-t_0)
	\right]
\end{align}
が得られる。
右辺の被積分関数には
\begin{align}
	\delta\left( \frac{R}{c} - t + t' \right)
\end{align}
という項があり、$t_0 < t - R/c < t_1$であれば(今$t_1$は自由に取れる)、
教科書の通り
\begin{align}
	(\mbox{右辺})
	= \frac{1}{4\pi c^2} \int_{\partial V} \ddif S' \vec{n}' \cdot
	\left( \frac{\nabla' \psi(\vec{x'}, t')}{R}
	- \frac{\vec{R}}{R^3} \psi(\vec{x}', t') - \frac{\vec{R}}{cR^2}
	\pdif{\psi(\vec{x}', t')}{t'} \right)_{t' = t - \frac{R}{c}}
\end{align}
である。
したがってKirchhoffの積分表示
\begin{align}
	\psi(\vec{x}, t) = &\int_V \ddif^3 x'
	\left[ D_{\rm ret}(\vec{x} - \vec{x}', t-t_0) \pdif{\psi(\vec{x}', t)}{t_0}
	+ \pdif{}{t_0} D_{\rm ret}(\vec{x}
	- \vec{x}', t-t_0) \psi(\vec{x}', t_0) \right] \\
	&\quad + \frac{1}{4\pi} \int_{\partial V} \ddif S' \vec{n}' \cdot
	\left( \frac{\nabla' \psi(\vec{x'}, t')}{R}
	- \frac{\vec{R}}{R^3} \psi(\vec{x}', t') - \frac{\vec{R}}{cR^2}
	\pdif{\psi(\vec{x}', t')}{t'} \right)_{t' = t - \frac{R}{c}}
\end{align}
が得られる。
ただしこの表式のためには境界上で$t_0 < t - R/c$が必要であるから
\begin{align}
	|\vec{x} - \vec{x}'| < c(t-t_0) \quad (\vec{x}' \in \partial V)
\end{align}
でなければならない。
つまり点$\vec{x}$から見て境界よりも遠い領域が$V$内にない場合は
体積積分が消えることになる
(つまり$V$が有限の場合は必ず体積積分は消える?)。

と思ったが、無限の場合も無限遠点が境界に含まれ、
外の体積積分は寄与しないのでは?とも思ったり。
そう考えるならば、無限遠の境界を無視しているわけではなく、
計算すると寄与がなくなる必要があるなと思った。
ということで、8.7節の例題の解である$\psi$が無限遠でどうなっているか考えることにする。
ここでは$\psi$は導出されるものであるから、最初に無限遠の寄与を含めていないということは、
無限遠の境界で面積積分が0になるものを選んでいるということである。
つまり解もそうなっていなければならないはずである。
例題では
\begin{align}
	\psi(\vec{x}, t) = \frac{f(\theta)}{r}
	e^{i(kr - \omega t)}
\end{align}
であり、面積積分の被積分関数(の定数倍)は$\vec{R} = \vec{x} - \vec{x}'$として
\begin{align}
	&\vec{n}' \cdot \left( 
		\frac{\nabla'\psi(\vec{x}', t')}{R}
		- \frac{\vec{R}}{R^3} \psi(\vec{x}', t')
		- \frac{\vec{R}}{cR^2} \pdif{\psi(\vec{x}', t')}{t'}
	\right) \\
	=& \vec{n}' \left[
		\frac{1}{R} \left\{
			\left( ik \psi - \frac{1}{r'} \psi \right) \vec{e}_{r'}
			+ \left( \frac{1}{r'} 
			\pdif{f(\theta')}{\theta'} e^{i(kr - \omega t)} \right)
			\vec{e}_{\theta'}
		\right\}
	\right]
\end{align}

\section{境界値について}
また境界値問題については初めに、
$V$の境界$S$上での境界値付き波動方程式
\begin{gather}
	\left( \Delta - \frac{1}{c^2} \pdif{^2}{t^2} \right) \psi(\vec{x}, t) = 0
	\label{eq:wave} \\
	\psi(\vec{x}, t) = f(\vec{x}, t) \quad (\vec{x} \in S)
\end{gather}
について考える。
偏微分$\nabla$を超曲面$S$上での偏微分$\nabla'$を用いて
\begin{gather}
	\nabla = \vec{e}_n \pdif{}{n} + \nabla' \\
	\Delta = \pdif{^2}{n^2} + \Delta'
\end{gather}
のように分解すると、上の方程式の解はもちろん波動方程式\eqref{eq:wave}の解であるから、
$S$上において
\begin{align}
	\left( \Delta - \frac{1}{c^2} \pdif{^2}{t^2} \right) \psi(\vec{x}, t)
	&= \left( \Delta' - \frac{1}{c^2} \pdif{^2}{t^2} \right) \psi(\vec{x}, t)
	+ \pdif{^2}{n^2} \psi(\vec{x}, t) \\
	&= \left( \Delta' - \frac{1}{c^2} \pdif{^2}{t^2} \right) f(\vec{x}, t)
	+ \pdif{^2}{n^2} \psi(\vec{x}, t) \\
	&= 0
\end{align}
が成り立たなければならない。
この第1項はある与えられた関数であり、
$\partial \psi / \partial n$で考えればこれはただの1階常微分方程式となるから、
与えられた$f$によって解が決定する
(\textcolor{red}{1点の情報しかないし微妙な気もする})。
またすでに$\nabla' \psi$の値は確定しているから、$\nabla \psi$の値は確定する。
次に境界値問題
\begin{gather}
	\left( \Delta - \frac{1}{c^2} \pdif{^2}{t^2} \right) \psi(\vec{x}, t) = 0 \\
	\nabla \psi(\vec{x}, t) = \vec{g}(\vec{x}, t) \quad (\vec{x} \in S)
\end{gather}
を考えると、$\psi$は$S$上では方程式
\begin{align}
	\nabla' \psi(\vec{x}, t) = \vec{g}_{\parallel} (\vec{x}, t)
\end{align}
を満たす必要があり、恐らく$S$が閉じていることから
定数分の不定性を除いて$S$常で$\psi$が決定されるのではないかと想像される。

\end{document}