\documentclass[a4paper, 10pt]{jsarticle}

% 余白
\usepackage[top=20truemm, bottom=25truemm, left=22truemm, right=22truemm, driver=dvipdfm, truedimen, margin=2cm]{geometry}
% 数式
\usepackage{amsmath, amssymb, amsthm}
\theoremstyle{definition}
\usepackage{ascmac}
\usepackage{mathtools}
\mathtoolsset{showonlyrefs,showmanualtags} 	 % 相互参照した式のみに番号を振る
% 画像
\usepackage[dvipdfmx]{graphicx}
\usepackage[subrefformat=parens]{subcaption}
\captionsetup{compatibility=false}
% ハイパーリンク
\usepackage[dvipdfmx, bookmarksnumbered]{hyperref}
\usepackage{pxjahyper}
\hypersetup{colorlinks=true, linkcolor=black, citecolor=black, urlcolor=black}

% コマンド定義
\def\vec#1{\mbox{\boldmath $#1$}}
\newcommand{\dif}[2]{\frac{{\rm d} #1}{{\rm d} #2}}
\newcommand{\pdif}[2]{\frac{\partial #1}{\partial #2}}
\newcommand{\ddif}{{\rm d}}
\DeclareMathOperator{\Div}{div}
\DeclareMathOperator{\Grad}{grad}
\DeclareMathOperator{\Rot}{rot}
\allowdisplaybreaks[4] 	 % 数式等のページ分割をさせる

\renewcommand{\thesubsubsection}{(\alph{subsubsection})}

\title{相互作用するラグランジアンについての考察}
\author{}

\begin{document}
\maketitle

\setcounter{section}{1}

\subsection{Lagrange形式}
今節では相対論的な系において、
場・粒子が存在してそれらが相互作用する場合を考える。
前節と同様に、場がある場合においても作用が存在して、
それを停留させる場、粒子の位置が実現すると考える。

$M$個の場$\{A_i\}_{i=1}^{M}$、$N$個の粒子$\{z_i\}_{i=1}^N$が存在する系を
考える。
場は各世界点$x$に依るため、$A_i = A_i(x)$と書ける。
粒子の位置$z_i$については少し難しく、相対論的に扱うためには、
位置$\vec{z}_i$と時刻$t_i$を同等に扱わなければならない。
そのため全粒子に世界点の移動を表すためのパラメータとして$\lambda_i$を導入し、
$z_i = z_i (\lambda_i)$としておく。
すなわち、$\{A_i\}_{i=1}^{M}$をまとめて$A$、
$\{z_i\}_{i=1}^N$をまとめて$z$と書くことにすると、作用は
\begin{align}
	S = S[A(x), z(\lambda)]
\end{align}
と書けることになる。

最終的には場と粒子が相互作用する系の作用を考えたいが、
相互作用が存在する場合であっても、それぞれ単独の作用が消えるわけではないから、
まずそれぞれ単独の作用を考える。

\subsubsection{場が単独の項}
場が単独の系の作用$S_\textrm{F}$は、場$A$が$x$に依るため
ラグランジアン密度$\mathcal{L}_\textrm{F}$を用いて
\begin{align}
	S_\textrm{F} = \int \ddif^4 x \mathcal{L}_\textrm{F}
\end{align}
と書けるだろう。
ここで$\mathcal{L}_\textrm{F}$の依存性は、
場$A$の物理がその1階微分までで決まることから、
\begin{align}
	\mathcal{L}_\textrm{F} = \mathcal{L}_\textrm{F} \left(
		A, \partial A, x
	\right)
\end{align}
となることが分かる。
ここから変分などを考えるが、場$\{A_i\}_i$はそれぞれ独立であるから、
添字を省略して考える。
こう書けるときに場$A$の変分を考えると、
\begin{align}
	\delta S_\textrm{F} &= \int \ddif^4 x \left[
		\mathcal{L}_\textrm{F} ( A + \delta A,
		\partial_\mu A + \delta \partial_\mu A, x)
		- \mathcal{L}_\textrm{F} ( A , \partial_\mu A, x)
	\right] \\
	&= \int \ddif^4 x \left[
		\pdif{\mathcal{L}_\textrm{F}}{A} \delta A
		+ \pdif{\mathcal{L}_\textrm{F}}{\left( \partial_\mu A \right)}
		\partial_\mu \delta A + O(\delta^2)
		\right] \\
		&= \int \ddif^4 x \left[
			\pdif{\mathcal{L}_\textrm{F}}{A} \delta A
			- \partial_\mu \pdif{\mathcal{L}_\textrm{F}}
			{\left( \partial_\mu A \right)} \delta A
		\right]
		+ \int \ddif S_\mu \pdif{\mathcal{L}_\textrm{F}}
		{\left( \partial_\mu A \right)} \delta A
		+ O(\delta^2) \\
		&= \int \ddif^4 x \left[
			\pdif{\mathcal{L}_\textrm{F}}{A}
			- \partial_\mu \pdif{\mathcal{L}_\textrm{F}}
			{\left( \partial_\mu A \right)} 
		\right] \delta A
		+ O(\delta^2) \quad
		\left( \because \text{境界において} \delta A = 0\right)
\end{align}
であるから、作用$S_\textrm{F}$が停留する条件
\begin{align}
	\pdif{\mathcal{L}_\textrm{F}}{A}
	- \partial_\mu \pdif{\mathcal{L}_\textrm{F}}
	{\left( \partial_\mu A \right)}
	= 0
\end{align}
を得る。
これが場のEuler-Lagrange方程式である。
さてここでNewton力学において粒子のラグランジアンに$\ddif f(\vec{z}, t) / \ddif t$
の不定性があったことを思い出すと、
同様に場のラグランジアンにおいても
\begin{align}
	\partial_\mu J^\mu \left( A, x \right)
\end{align}
の不定性があることが分かる。
なぜならこのラグランジアンにおける場の変分の1次は、第2項がないときと比べて
\begin{align}
	\int \ddif^4 x \ \partial_\mu \left[ \pdif{}{A} \left( J^\mu
	\left( A, x \right) \right) \delta A \right]
	= \int \ddif S_\mu \ \pdif{}{A} \left( J^\mu \left( A, x \right)
	\right) \delta A
\end{align}
しか違わないが、これは境界で積分するため0となるからである。
注意点は、今は相対論的な系を考えているから、
ラグランジアンはLorentz変換に対して不変(もしくはスカラー)で
なければならないということである。
したがって、添字は潰す必要がある。
今後ラグランジアンの不変性を考える場合には、
この不定性を除いて考えなければならないが、わざわざ言及しないこととする。

ここまでを踏まえて、対称性からある程度ラグランジアンの形を決める。
並進対称性はあるはずであるから、
$x \to x + a$としたときにラグランジアンは変わらないはずである。
そのためにはラグランジアン$\mathcal{L}_\textrm{F}$は$x$に依存していなければよい。
つまり
\begin{align}
	\mathcal{L}_\textrm{F} = \mathcal{L}_\textrm{F} \left(
		A, \partial_\mu A
	\right)
\end{align}
と書ければよい。

次に具体的に今回の場合として具体的に電磁場を考える。
電磁場は4元ベクトル$A_\mu$を使って表される。
また電磁場$A_\mu$はゲージ変換、すなわち任意関数$\chi$を用いて
\begin{align}
	A \to A + \partial \chi
	\label{eq:GaugeTrans}
\end{align}
と変換したときに、物理が不変となる。
したがって変換\eqref{eq:GaugeTrans}の下でラグランジアンが
不変とならなければならない。
このために新たに場の強さと呼ばれる量
\begin{align}
	F_{\mu \nu} = \partial_\mu A_\nu - \partial_\mu A_\nu
\end{align}
を導入する。

\subsubsection{相互作用項}
ラグランジアン密度を
\begin{align}
	\mathcal{L} =
	\mathcal{L} \left( z, \dot{z}, A, \partial_\nu A, x \right)
\end{align}
と書く。






$\partial_\mu j^\mu$という形で書かれるはずである。
一般の$a$に対して考えるのは難しいため、
まず微小量$\delta a$の場合の1次の変化を見ることにする。
すると
\begin{align}
	\delta \mathcal{L}_\textrm{F}
	&= \pdif{\mathcal{L}_\textrm{F}}{A} \pdif{A}{x_\nu} \delta a^\mu
	+ \pdif{\mathcal{L}_\textrm{F}}{\left( \partial_\nu A \right)}
	\pdif{}{x_\nu} \left( \partial_\nu A \right) \delta a^\rho
	\pdif{\mathcal{L}_\textrm{F}}{\left( \partial_\nu A \right)}
	+ \pdif{\mathcal{L}_\textrm{F}}{x^\mu} \delta a^\mu \\
	&\stackrel{\textrm{(E-L eq.)}}{=} %\overset{\textrm{(E-L eq.)}}{=}
	\partial_\nu
	\pdif{\mathcal{L}_\textrm{F}}{\left( \partial_\nu A \right)}
	\partial_\nu A \delta a^\rho
	+ \pdif{\mathcal{L}_\textrm{F}}{\left( \partial_\nu A \right)}
	\partial_\nu \partial_\nu A \delta a^\rho
	+ \pdif{\mathcal{L}_\textrm{F}}{x^\mu} \delta a^\mu \\
	&= \partial_\nu \left[
		\pdif{\mathcal{L}_\textrm{F}}{\left( \partial_\nu A \right)}
		\partial_\nu A \delta a^\rho
	\right] + \pdif{\mathcal{L}_\textrm{F}}{x^\mu} \delta a^\mu
\end{align}
となることが分かる。
この第1項はラグランジアンの不定性で消せるから、
$\pdif{\mathcal{L}_\textrm{F}}{x^\mu} = 0$でなければならない。
すなわち$\mathcal{L}_\textrm{F}$に$x$依存性はない。
このようにすれば、微小量$\delta a$だけではなく一般の$a$についても
ラグランジアンが不定であることが分かる。

\end{document}