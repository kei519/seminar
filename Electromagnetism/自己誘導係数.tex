\documentclass[a4paper, 12pt]{jsarticle}

% 余白
\usepackage[top=20truemm, bottom=25truemm, left=22truemm, right=22truemm, driver=dvipdfm, truedimen, margin=2cm]{geometry}
%\allowdisplaybreaks[4] 	 % 数式等のページ分割をさせる
% 数式
\usepackage{amsmath, amssymb, amsthm}
\theoremstyle{definition}
\usepackage{ascmac}
\usepackage{mathtools}
\mathtoolsset{showonlyrefs,showmanualtags} 	 % 相互参照した式のみに番号を振る
% 画像
\usepackage[dvipdfmx]{graphicx}
\usepackage[subrefformat=parens]{subcaption}
\captionsetup{compatibility=false}
% ハイパーリンク
\usepackage[dvipdfmx, bookmarksnumbered]{hyperref}
\usepackage{pxjahyper}
\hypersetup{colorlinks=true, linkcolor=black, citecolor=black, urlcolor=black}

% コマンド定義
\def\vec#1{\mbox{\boldmath $#1$}}
\newcommand{\dif}[2]{\frac{{\rm d} #1}{{\rm d} #2}}
\newcommand{\pdif}[2]{\frac{\partial #1}{\partial #2}}
\newcommand{\ddif}{{\rm d}}
\DeclareMathOperator{\Div}{div}
\DeclareMathOperator{\Grad}{grad}
\DeclareMathOperator{\Rot}{rot}

\title{円形断面導線の自己誘導係数}

\begin{document}
\maketitle
\allowdisplaybreaks[4]

断面が円形でその半径が$a$である導線が長さ$l$の閉回路をなしているとき、
その自己誘導係数を求めることを考える。
ここで$a \gg l$、導線内の透磁率は$\mu$であるとする。
このときその自己誘導係数$L$を、
導線内部の自己誘導係数$L_i$と導線に囲まれた外部の自己誘導係数$L_e$に分けて考える。

まずはじめに$L_i$を求める。
電流$I$が流れている場合、
その分布が一様であるとき導線内において電流密度$\vec{i}_e = I/\pi a^2$となる。
また$a \ll l$であるから局所的には円柱とみなせるから、
円柱座標でのPoisson方程式
\begin{align}
    \Delta \vec{A}
    = \frac{1}{r} \pdif{}{r} \left( r \pdif{\vec{A}}{r} \right)
    = -\mu \vec{i}_e
    = -\mu \frac{I}{\pi a^2} \vec{e}_z
\end{align}
より
\begin{align}
    A_z = -\frac{\mu I}{4\pi} \left( \frac{r}{a} \right)^2
\end{align}
となる。
したがって
\begin{align}
    \vec{B} = \Rot \vec{A}
    = -\pdif{A_z}{r} \vec{e}_\theta
    = \frac{\mu I}{2 \pi a^2} r
\end{align}
であるから導線内の磁場のエネルギーは
\begin{align}
    W_m = \frac{1}{2 \mu} \int_V \vec{B}^2 \ddif^3 x
    = \frac{I^2}{2} \frac{\mu l}{8 \pi}
\end{align}
となり、この近似のもとで
\begin{align}
    L_i \simeq \frac{\mu l}{8\pi}
\end{align}
である。

次に$L_e$を求める。
閉曲線が作る面を通る磁束$\Phi$は導線の内側の側面に沿った経路$C'$で
\begin{align}
    \Phi = \int_C' \vec{A} \cdot \ddif \vec{r}
\end{align}
と積分することで求められる。
この積分で用いられる$\vec{A}$は導線の外のベクトルポテンシャルであるから、
電流は導線の軸にのみ流れていると考えることができる。
このように考えるとき、
軸上で$\vec{A}$は発散するが曲線の長さは側面と軸上でほぼ変わらないから、
側面でのベクトルポテンシャル$\vec{A}'$と、
それを導線の軸上、つまり導線を線とみなした経路$C$での積分で近似し
\begin{align}
    \Phi \simeq \int_C \vec{A}' \cdot \ddif \vec{r}
\end{align}
とみなせる。

次に$\vec{A}'$を近似で求めることを考える。
これは
\begin{align}
    \vec{A}' (\vec{r}')
    = \frac{\mu_0 I}{4\pi} \int_C \frac{\ddif \vec{r}}{|\vec{r} - \vec{r}'|}
\end{align}
で求められる。
ここで導線を線とみなし$\vec{r}$と$\vec{r}'$の距離$R$、
$a \ll \Delta \ll l$を満たす$\Delta$を導入し、
積分を$R < \Delta, R > \Delta$に分けて積分する。
まず$R > \Delta \gg a$では$|\vec{r} - \vec{r}'| \simeq R$とみなせるから、
\begin{align}
    \int_{R > \Delta} \frac{\ddif \vec{r}}{|\vec{r} - \vec{r}'|}
    \simeq \int_{R > \Delta} \frac{\ddif \vec{r}}{R}
\end{align}
と書ける。
$R < \Delta \ll l$では$\ddif \vec{r}$は全領域に渡って平行だとみなせ、
それを$\vec{t}$と書くことにすると、
\begin{align}
    \int_{R < \Delta} \frac{\ddif \vec{r}}{|\vec{r} - \vec{r}'|}
    &\simeq \vec{t}
    \int_{-\Delta}^{\Delta} \frac{\ddif x}{\sqrt{a^2 + x^2}} \\
    &= 2\vec{t}
    \log \left[ \frac{\Delta}{a} + \sqrt{\left( \frac{\Delta}{a} \right)^2 + 1} \right] \\
    &\simeq 2\vec{t} \log \frac{2\Delta}{a} \qquad
    \left( \because a \ll \Delta \right) \\
    &\simeq \int_{a/2 < R < \Delta} \frac{\ddif \vec{r}}{R}
\end{align}
が成り立つ。
以上をまとめると、
\begin{align}
    \vec{A}' \simeq \frac{\mu_0 I}{4\pi} \int_{R > a/2} \frac{\ddif \vec{r}}{R}
\end{align}
と書け、
\begin{align}
    \Phi \simeq \frac{\mu_0 I}{4\pi} \int_C \int_{R > a/2}
    \frac{\ddif \vec{r} \cdot \ddif \vec{r}'}{|\vec{r} - \vec{r}'|}
\end{align}
つまり
\begin{align}
    L_e \simeq \frac{\mu_0}{4\pi} \int_C \int_{R > a/2}
    \frac{\ddif \vec{r} \cdot \ddif \vec{r}'}{|\vec{r} - \vec{r}'|}
\end{align}
が得られる。

\end{document}