\documentclass[a4paper]{jsarticle}

% 余白
\usepackage[top=20truemm, bottom=25truemm, left=22truemm, right=22truemm]{geometry}
% 数式
\usepackage{amsmath, amssymb}
\usepackage{ascmac}
\usepackage{mathtools}
\mathtoolsset{showonlyrefs,showmanualtags} 	 % 相互参照した式のみに番号を振る
% 画像
\usepackage[dvipdfmx]{graphicx}
\usepackage[subrefformat=parens]{subcaption}
\captionsetup{compatibility=false}
% ハイパーリンク
\usepackage[dvipdfmx]{hyperref}
\usepackage{pxjahyper}

% コマンド定義
\def\vec#1{\mbox{\boldmath $#1$}}
\newcommand{\dif}[2]{\frac{{\rm d} #1}{{\rm d} #2}}
\newcommand{\pdif}[2]{\frac{\partial #1}{\partial #2}}
\newcommand{\ddif}{{\rm d}}

\title{4.4b \ Gravitational Radiation}

\begin{document}
\maketitle

Coulombの静電磁場から、Maxwellの電磁気学に移ったときの一番大きな変化は、全体が動的になったことだった。
またそれによって、電磁波が伝搬することができる。
Newtonの万有引力から、Einsteinの一般相対論に移ったときにも同じことが、つまり重力波が存在することが分かるようになる。
線形近似した重力波は、ソースフリーな線形化Einstein方程式(4.4.11)、(4.4.12)から
\begin{align}
	\partial^a \overline{\gamma}_{ab} &= 0 \tag{4.4.25} \\
	\partial^c \partial_c \overline{\gamma}_{ab} &= -16 \pi T_{ab} = 0 \tag{4.4.26}
\end{align}
に従っている。

最初に式(4.4.25)を満たすようにゲージを選んだが、それでも更に$\gamma_{ab} \to \gamma_{ab} + \partial_a \xi_b + \partial_b \xi_a$のゲージ変換を施すことを考える。
元の$\gamma^{\prime}_{ab}$から$\gamma_{ab} = \gamma^{\prime}_{ab} + \partial_a \xi_b + \partial_b \xi_a$に変更すると、
\begin{align}
	\gamma &= \eta^{ab} \gamma_{ab} \\
	&= \eta^{ab} \left( \gamma^{\prime}_{ab} + \partial_a \xi_b + \partial_b \xi_a \right) \\
	&= \gamma^{\prime} + 2\partial^a \xi_a \label{eq:gamma} \\
	\overline{\gamma}_{ab} &= \gamma_{ab} - \frac{1}{2} \gamma \\
	&= \gamma^{\prime}_{ab} + \partial_a \xi_b + \partial_b \xi_a - \frac{1}{2} \eta_{ab} \left( \gamma^{\prime} + 2\partial^c \xi_c \right) \\
	&= \overline{\gamma}^{\prime}_{ab} + \partial_a \xi_b + \partial_b \xi_a - \eta_{ab} \partial^c \xi_c \label{eq:gammaBar} \\
	\partial^a \overline{\gamma}_{ab} &= \partial^a \overline{\gamma}^{\prime}_{ab} + \partial^a \partial_a \xi_b + \partial^a \partial_b \xi_a - \partial^a \eta_{ab} \partial^c \xi_c \\
	&= \partial^a \overline{\gamma}^{\prime}_{ab} + \partial^a \partial_a \xi_b + \partial^a \partial_b \xi_a - \partial_b \partial^c \xi_c \\
	&= \partial^a \overline{\gamma}^{\prime}_{ab} + \partial^a \partial_a \xi_b \label{eq:pgamma}
\end{align}
となる。
つまり、
\begin{align}
	\partial^b \partial_b \xi_a = 0 \tag{4.4.27}
\end{align}
を満たす$\xi^a$によるゲージ変換の自由度は残る。
これはLorenzゲージが、ベクトルポテンシャル$A_a$を唯一定めるわけではなく、
\begin{align}
	\partial^a \partial_a \chi = 0 \tag{4.4.28}
\end{align}
を満たす、$\chi$による制限されたゲージ自由度$A_a \to A_a + \partial^a \chi$が許されることと似ている。
電磁波を考察するには、ゲージ自由度を自由空間($j_a = 0$)において、ある慣性系で(4.4.28)を満たしつつ、$A_0 = 0$となるように選ぶのが便利である。
これはCoulombゲージ、もしくは放射ゲージと呼ばれている。
これは以下のようにすると得られる。
選んだ慣性系の$t=t_0$表面において、
\begin{align}
	\nabla^2 \chi = - \nabla \cdot \vec{A} \\
	\pdif{\chi}{t} = -A_0
\end{align}
が満たされるように$\chi$を選ぶ(式(4.4.25)の解は、$\chi$、$\partial \chi / \partial t$の任意の初期条件でただ1つ解が定まることが、10.1節から分かる)。
すると、関数
\begin{align}
	f = A_0 + \pdif{\chi}{t}
\end{align}
は、
\begin{align}
	\partial^a \partial_a A_b = -4\pi j_b \tag{4.2.32}
\end{align}
と式(4.4.28)より、
\begin{align}
	\partial^a \partial_a f = \partial^a \partial_a A_0 + \pdif{}{t} \partial^a \partial_a \chi = -4 \pi j_0 \tag{4.4.31}
\end{align}
を満たし、$t=t_0$では
\begin{align}
	f &= 0 \\
	\pdif{f}{t} = \pdif{A_0}{t} + \pdif{^2 \chi}{t^2}
	&= \nabla \cdot \vec{A} + \nabla^2 \chi = 0
\end{align}
も満たすことが分かる。
考えている範囲(もっと正確に言うと、$t=t_0$から円錐の各点までの間に)にソースが無ければ、方程式(4.4.31)の解は$f=0$となる。
したがって、この解を用いてゲージ変換$A_a \to A_a + \partial_a \chi$を行えば、Lorenzゲージが得られる。

同じように、制限されたゲージ自由度(4.4.27)を満たしつつ、ソースフリー($T_{ab}=0$)な領域において、$\gamma=0$、$\gamma_{0\mu}=0 (\mu=1,2,3)$となるような、線形化された重力場を用いることにする。
また、もし全時空においてソースが無ければ、$\gamma_{00}=0$となるようにし、無限遠で良い振る舞いをするように選ぶ。
そのために、$t=t_0$において
\begin{align}
	2 \left( -\pdif{\xi_0}{t} + \nabla \cdot \vec{\xi} \right) &= -\gamma \tag{4.4.34a} \\
	2 \left[ -\nabla^2 \xi_0 + \nabla \cdot \left( \pdif{\vec{\xi}}{t} \right)\right] &= -\pdif{\gamma}{t} \tag{7.4.34b} \\
	\pdif{\xi_{\mu}}{t} + \pdif{\xi_0}{x^{\mu}} &= -\gamma_{0\mu} \quad (\mu=1, 2, 3) \tag{4.4.34c} \\
	\nabla^2 \xi_{\mu} + \pdif{}{x^{\mu}} \left(\pdif{\xi_0}{t}\right) &= -\pdif{\gamma_{0\mu}}{t} \quad (\mu = 1, 2, 3) \tag{4.4.34d}
\end{align}
を満たすように$\xi_{\mu}$を選ぶ。
上で電磁気学においてやったのと同じことをすると、$\gamma=0$、$\gamma_{0\mu}=0$とできることが示せる。

関数$f$を
\begin{align}
	f = \gamma + 2 \partial^a \xi_a
\end{align}
とすると、これは
\begin{align}
	\overline{\gamma} = {\overline{\gamma}^a}_a
	= {\gamma^a}_a - \frac{1}{2} {\delta^a}_b \gamma
	= \frac{1}{2} \gamma
\end{align}
であることを用いると、(4.4.26)と(4.4.27)より
\begin{align}
	\partial^a \partial_a f &= \partial^a \partial_a \gamma + 2 \partial^a \partial_a \partial^b \xi_b \\
	&= 2\partial^a \partial_a \overline{\gamma} + 2 \partial^b \partial^a \partial_a \xi_b \\
	&= -32 \pi {T^a}_a \\
	&= 0
\end{align}
また$t=t_0$で
\begin{align}
	f &= 0 \\
	\pdif{f}{t} &= \pdif{\gamma}{t} + 2 \pdif{}{t} ( \partial^{\mu} \xi_{\mu} ) \\
	&= -2 \left( -\nabla^2 \xi_0 + \nabla \cdot \pdif{\vec{\xi}}{t} \right) + 2 \pdif{}{t} \left( -\pdif{\xi_0}{t} + \nabla \cdot \vec{\xi} \right) \\
	&= 2 \left( \pdif{^2 \xi_0}{t^2} + \nabla^2 \xi_0 \right) \\
	&= 0
\end{align}
となるから、ソースが無ければ、$f=0$となる。

関数$g_{\mu} (\mu = 1, 2, 3)$を
\begin{align}
	g_{\mu} &= \gamma_{0\mu} + \partial_0 \xi_{\mu} + \partial_{\mu} \xi_0 \\
	&= \gamma_{0\mu} + \pdif{\xi_{\mu}}{t} + \pdif{\xi_0}{x^{\mu}}
\end{align}
で定義すると、先と同様に
\begin{align}
	\partial_{\mu} \partial^{\mu} g_{\nu} = 0
\end{align}
となる。
また、$t=t_0$においては
\begin{align}
	g_{\mu} &= 0 \\
	\pdif{g_{\mu}}{t} &= \pdif{\gamma_{0\mu}}{t} + \pdif{^2 \xi_{\mu}}{t^2} + \pdif{}{t} \left( \pdif{\xi_0}{x^{\mu}} \right) \\
	&= \pdif{\gamma_{0\mu}}{t} + \nabla^2 \vec{\xi_{\mu}} + \pdif{}{x^{\mu}} \left( \pdif{\xi_0}{t} \right) \\
	&= 0
\end{align}
であるから、これも同様にソースがなければ、$g_{\mu}=0$となる。
したがって、この$\xi_{\mu}$を選ぶと、$\gamma$、$\gamma_{0\mu} (\mu = 1, 2, 3)$が0になる。

$\gamma_{00}$については、$\gamma=0$より、$\gamma_{ab} = \overline{\gamma}_{ab}$となるから、式(4.4.25)より
\begin{align}
	\partial^{\mu} \gamma_{\mu 0} = \pdif{\gamma_{00}}{t} = 0 \tag{4.4.35}
\end{align}
が得られる。
したがって、線形化されたEinstein方程式(4.4.26)を用いると、全時空においてソースがないことから、
\begin{align}
	\partial^c \partial_c \gamma_{00} = \nabla^2 \gamma_{00} = -16 \pi T_{00} = 0 \tag{4.4.36}
\end{align}
が得られる。
この解のうち、無限遠方で振る舞いが良いのは、$\gamma_{00} = {\rm const}$だけである。
ここで、更にゲージ変換を行うと、$\gamma_{00}=0$とできる。

この放射ゲージを使って、ソースフリーな場合の線形化Einstein方程式の解を求める。
定テンソル場$H_{ab}$を用いて表される、平面波
\begin{align}
	\gamma_{ab} = H_{ab} \exp \left( i \sum_{\mu = o}^3 k_{\mu} x^{\mu} \right)
\end{align}
は、ゲージ条件(4.4.26)を
\begin{align}
	\sum_{\mu} k^{\mu} k_{\mu} = 0
\end{align}
のときのみ満たす。
またゲージ条件(4.4.25)より
\begin{align}
	\partial^{\mu} \gamma_{\mu\nu} &= \partial^{\mu} H_{\mu\nu} \exp \left( i \sum_{\rho = o}^3 k_{\rho} x^{\rho} \right) \\
	&= i k^{\rho} {\delta^{\mu}}_{\rho} H_{\mu\nu} \exp \left( i \sum_{\rho = o}^3 k_{\rho} x^{\rho} \right) = 0 \\
	k^{\mu} H_{\mu\nu} &= 0 \tag{4.4.39a}
\end{align}
$\gamma_{0\mu}$より
\begin{align}
	\gamma_{0\mu} &= H_{0\mu} \exp \left( i \sum_{\rho = o}^3 k_{\rho} x^{\rho} \right) = 0 \\
	H_{0\mu} = 0 \tag{4.4.39b}
\end{align}
$\gamma = 0$より
\begin{align}
	\gamma &= \sum_{\mu=0}^3 {H^{\mu}}_{\mu} \exp \left( i \sum_{\rho = o}^3 k_{\rho} x^{\rho} \right) = 0 \\
	\sum_{\mu=0}^3 {H^{\mu}}_{\mu} &= 0 \tag{4.4.39c}
\end{align}
を得る。
これは全部で9個の式だが、(4.4.39b)が満たされれば、(4.4.39a)の$\nu=0$も満たされるから、独立した式は8つである。
$H_{\mu\nu}$の自由度は10だから、2つが残っていて、$H_{\mu\nu}$は線形独立な解を2つ持つ。
この2つは、重力波の独立した偏極状態を表している。
真空中の振る舞いの良い、線形化Einstein方程式の任意解はこの平面波の重ね合わせで書ける。

重力波を検出するもっとも単純なものは、2つの物質の相対加速度、つまり潮汐力を見ることである。
2つの自由落下している物体の相対加速度は測地線偏差方程式(3.3.18)によって決まる。
\begin{align}
	a^{a} = -{R_{cbd}}^a X^b T^c T^d \tag{3.3.18}
\end{align}
今の場合では、2つの物質が選んだ慣性系において、ほとんど静止していれば、$T^a$はほとんど時間軸と平行なベクトルであるから、
\begin{align}
	\dif{^2 X^{\mu}}{t^2} \approx \sum_{\nu} {R_{\nu00}}^{\mu} X^{\nu} \tag{4.4.40}
\end{align}
となる。
放射ゲージでは、
\begin{align}
	{R_{\mu\nu\rho}}^{\sigma} = \pdif{}{x^{\nu}} {\Gamma^{\sigma}}_{\mu\rho} - \pdif{}{x^{\mu}} {\Gamma^{\sigma}}_{\nu\rho} + \sum_{\alpha} \left( {\Gamma^{\alpha}}_{\mu\rho} {\Gamma^{\sigma}}_{\alpha\nu} - {\Gamma^{\alpha}}_{\nu\rho} {\Gamma^{\sigma}}_{\alpha\mu} \right) \tag{3.4.4}
\end{align}
を用いると、
\begin{align}
	{\Gamma^{\mu}}_{\nu\rho} &= \sum_{\alpha} \frac{1}{2} g^{\mu\alpha} \left( \partial_{\nu} g_{\rho\alpha} + \partial_{\rho} g_{\nu\alpha} - \partial_{\alpha} g_{\nu\rho} \right) \\
	&\simeq \sum_{\alpha} \frac{1}{2} \eta^{\mu\alpha} \left( \partial_{\nu} \gamma_{\rho\alpha} + \partial_{\rho} \gamma_{\nu\alpha} - \partial_{\alpha} \gamma_{\nu\rho} \right) \\
	{\Gamma^0}_{\mu\nu} &= \frac{1}{2} \pdif{\gamma_{\mu\nu}}{t} \\
	{\Gamma^{\mu}}_{\nu 0} &= \frac{1}{2} \pdif{{\gamma^{\mu}}_{\nu}}{t}
\end{align}
より、線形化Riemanテンソル
\begin{align}
	{R_{\nu 00}}^{\mu} &= \pdif{}{t} \left( \frac{1}{2} \pdif{{\gamma^{\mu}}_{\nu}}{t} \right) + \sum_{\alpha} \frac{1}{2} \pdif{{\gamma^{\alpha}}_{\nu}}{t} \cdot \frac{1}{2} \pdif{{\gamma^{\mu}}_{\alpha}}{t} \\
	&= \frac{1}{2} \pdif{^2 {\gamma^{\mu}}_{\nu}}{t^2} + ?
\end{align}
を得られる。
ここから分かることは、先程出した平面波解は、曲率が0でないため、ゲージ変換をしても消すことができない、つまり意味のあるものになっているということである。
この計測は、支柱からぶら下げられた2つの物質が離れる様子を、正確に観測することで行うことができる。
また代わりに、物質が自由運動できるわけではないが、固体に結びつけると、その固体に重力の干潮力による圧力がかかる。
この重力波によって周期的な圧力が加わると、もし固体の共振振動数が重力波の振動数と近ければ、固体も振動し、それを観測することができる(考案者:Joseph Weber)。
観測可能な振動数についての$\gamma_{\mu\nu}$は$10^{-17}$以上にはならないと考えられている。
これは2つの物質間の距離変化比$\Delta X /X$が$10^{-17}$を超えることはないということを意味する。
例えば2物質間の距離が1 m ならば、核の直径の1/100程度しか動かない
したがって、この観測には非常に繊細に行わなければならない。

\end{document}