\documentclass[a4paper, 12pt]{jsarticle}

% 余白
\usepackage[top=20truemm, bottom=25truemm, left=22truemm, right=22truemm, driver=dvipdfm, truedimen, margin=2cm]{geometry}
% 数式
\usepackage{amsmath, amssymb, amsthm}
\theoremstyle{definition}
\usepackage{ascmac}
\usepackage{mathtools}
\mathtoolsset{showonlyrefs,showmanualtags} 	 % 相互参照した式のみに番号を振る
% 画像
\usepackage[dvipdfmx]{graphicx}
\usepackage[subrefformat=parens]{subcaption}
\captionsetup{compatibility=false}
% ハイパーリンク
\usepackage[dvipdfmx, bookmarksnumbered]{hyperref}
\usepackage{pxjahyper}
\hypersetup{colorlinks=true, linkcolor=black, citecolor=black, urlcolor=black}

% コマンド定義
\def\vec#1{\mbox{\boldmath $#1$}}
\newcommand{\dif}[2]{\frac{{\rm d} #1}{{\rm d} #2}}
\newcommand{\pdif}[2]{\frac{\partial #1}{\partial #2}}
\newcommand{\ddif}{{\rm d}}
\DeclareMathOperator{\Div}{div}
\DeclareMathOperator{\Grad}{grad}
\DeclareMathOperator{\Rot}{rot}
\newtheorem*{theorem}{定理}
\newtheorem*{definition}{定義}
\newtheorem*{lemma}{補題}
\newtheorem*{exercise}{問題}
\newtheorem*{answer}{解答}
\renewcommand{\proofname}{証明}
\allowdisplaybreaks[4]

\title{レジュメ}

\begin{document}
\maketitle

\setcounter{section}{7}

\section{Mass, charge, and angular momentum}
\subsection{Charges in curved spacetime}
向き付可能な計量のあるn次元多様体で体積要素を
$\epsilon_{a_1 \cdots a_n}$と書くことにする。
すると
\begin{align}
	\epsilon^{a_1 \cdots a_p c_{p+1} \cdots c_n}
	\epsilon_{b_1 \cdots b_p c_{p+1} \cdots c_n}
	= \pm p! (n-p)! \delta^{a_1}_{[b_1} \cdots \delta^{a_p}_{b_p]}
	\tag{8.1} \label{8.1}
\end{align}
が示せる。
\begin{proof}
	\begin{align}
		\epsilon_{a_1 \cdots a_n} = \sqrt{|g|} \varepsilon_{a_1 \cdots a_n}
	\end{align}
	を用いれば、
	\begin{align}
		\epsilon^{\mu_1 \cdots \mu_n} \epsilon_{\nu_1 \cdots \nu_n}
		&= \varepsilon^{\mu_1 \cdots \mu_n} \varepsilon_{\nu_1 \cdots \nu_n} \\
		&= \det \left( \begin{array}{ccc}
			\delta^{\mu_1}_{\nu_1} & \cdots & \delta^{\mu_1}_{\nu_n} \\
			\vdots & \ddots & \vdots \\
			\delta^{\mu_n}_{\nu_1} & \cdots & \delta^{\mu_n}_{\nu_n}
		\end{array} \right)
	\end{align}
	を使って展開したりすれば良いんだろうな~。
\end{proof}
ここで$+$はRieman符号数、$-$はLorentz符号数に対応する。
\begin{definition}
	$p$形式$X$のホッジ双対$\star X$を$(n - p)$形式
	\begin{align}
		(\star X)_{a_1 \cdots a_{n-p}} = \frac{1}{p!}
		\epsilon_{a_1 \cdots a_{n - p} b_1 \cdots b_p} X^{b_1 \cdots b_p}
		\tag{8.2} \label{8.2}
	\end{align}
	で定義する。
\end{definition}
\begin{lemma}
	$p$形式$X$について
	\begin{gather}
		\star (\star X) = \pm (-1)^{p(n-p)} X \tag{8.3} \label{8.3} \\
		(\star \ddif \star X)_{a_1 \cdots a_{p-1}}
		= \pm (-1)^{p(n-p)} \nabla^b X_{a_1 \cdots a_{p-1} b}
		\tag{8.4} \label{8.4}
	\end{gather}
	が成り立つ。
	ここで符号は体積要素と同じ使い方である。
\end{lemma}
\begin{proof}
	\begin{align}
		&\quad \left[ \star \left( \star X \right) \right]_{\mu_1 \cdots \mu_p} \\
		&= \frac{1}{(n-p)!}
		\epsilon_{\mu_1 \cdots \mu_p \alpha_1 \cdots \alpha_{n-p}}
		\left( \star X \right)^{\alpha_1 \cdots \alpha_{n-p}} \\
		&= \frac{1}{(n-p)!} g^{\alpha_1 \nu_1} \cdots
		g^{\alpha_{n-p} \nu_{n-p}}
		\epsilon_{\mu_1 \cdots \mu_p \alpha_1 \cdots \alpha_{n-p}}
		\left( \star X \right)_{\nu_1 \cdots \nu_{n-p}} \\
		&= \frac{1}{(n-p)!} g^{\alpha_1 \nu_1} \cdots
		g^{\alpha_{n-p} \nu_{n-p}}
		\epsilon_{\mu_1 \cdots \mu_p \alpha_1 \cdots \alpha_{n-p}} \cdot
		\frac{1}{p!} \epsilon_{\nu_1 \cdots \nu_{n-p} \beta_1 \beta_p}
		X^{\beta_1 \cdots \beta_p} \\
		&= \frac{1}{p! (n-p)!} g^{\alpha_1 \nu_1} \cdots
		g^{\alpha_{n-p} \nu_{n-p}} g^{\beta_1 \gamma_1} \cdots
		g^{\beta_p \gamma_p}
		\epsilon_{\mu_1 \cdots \mu_p \alpha_1 \cdots \alpha_{n-p}}
		\epsilon_{\nu_1 \cdots \nu_{n-p} \beta_1 \cdots \beta_p}
		X_{\gamma_1 \cdots \gamma_p} \\
		&= \frac{1}{p! (n-p)!}
		\epsilon^{\alpha_1 \cdots \alpha_{n-p} \underbrace{\gamma_1 \cdots \gamma_p}_{p個を(n-p)個前に送る}}
		\epsilon_{\mu_1 \cdots \mu_p \alpha_1 \cdots \alpha_{n-p}}
		X_{\gamma_1 \cdots \gamma_p} \\
		&= \pm (-1)^{p (n-p)}
		\delta^{\gamma_1}_{[\mu_1} \cdots \delta^{\gamma_p}_{\mu_p]}
		X_{\gamma_1 \cdots \gamma_p} \\
		&= \pm (-1)^{p (n-p)} X_{[\mu_1 \cdots \mu_p]} \\
		&= \pm (-1)^{p (n-p)} X_{\mu_1 \cdots \mu_p}
	\end{align}
	\begin{align}
		&\quad \left[ \star \ddif \star X \right]_{\mu_1 \cdots \mu_{p-1}} \\
		&= \frac{1}{(n-p+1)!} 
		{\epsilon_{\mu_1 \cdots \mu_{p-1}}}^{\nu_1 \cdots \nu_{n-p+1}}
		\left( \ddif \star X \right)_{\nu_1 \cdots \nu_{n-p+1}} \\
		&= \frac{1}{(n-p+1)!}
		{\epsilon_{\mu_1 \cdots \mu_{p-1}}}^{\nu \nu_1 \cdots \nu_{n-p}} \cdot
		(n-p+1) \partial_\nu \left( \star X \right)_{\nu_1 \cdots \nu_{n-p}} \\
		&= \frac{1}{(n-p)!}
		{\epsilon_{\mu_1 \cdots \mu_{p-1}}}^{\nu \nu_1 \cdots \nu_{n-p}}
		\partial_\nu \left( \frac{1}{p!}
		\epsilon_{\nu_1 \cdots \nu_{n-p} \beta_1 \cdots \beta_{p}}
		X^{\beta_1 \cdots \beta_p} \right) \\
		&= \frac{(-1)^{p(n-p)}}{p! (n-p)!} g^{\alpha \nu}
		g_{\beta_1 \gamma_1} \cdots g_{\beta_p \gamma_p}
		{\epsilon^{\gamma_1 \cdots \gamma_{p}}}_{\nu_1 \cdots \nu_{n-p}}
		{\epsilon_{\mu_1 \cdots \mu_{p-1} \alpha}}^{\nu_1 \cdots \nu_{n-p}}
		\frac{1}{\sqrt{|g|}}
		\partial_\nu \left( \sqrt{|g|} X^{\beta_1 \cdots \beta_p} \right) \\
		&= \pm (-1)^{p(n-p)} g^{\alpha \nu}
		g_{\beta_1 \gamma_1} \cdots g_{\beta_p \gamma_p}
		\delta^{\gamma_1}_{[\mu_1} \cdots \delta^{\gamma_{p-1}}_{\mu_{p-1}}
		\delta^{\gamma_p}_{\alpha]} \frac{1}{\sqrt{|g|}}
		\partial_\nu \left( \sqrt{|g|} X^{\beta_1 \cdots \beta_p} \right) \\
		&= \pm (-1)^{p(n-p)}
		g_{\beta_1 \gamma_1} \cdots g_{\beta_{p-1} \gamma_{p-1}}
		\delta^{\gamma_1}_{[\mu_1} \cdots \delta^{\gamma_{p-1}}_{\mu_{p-1}}
		\frac{1}{\sqrt{|g|}} \partial_{\beta_p]}
		\left( \sqrt{|g|} X^{\beta_1 \cdots \beta_p} \right) \\
		&= \pm (-1)^{p(n-p)}
		g_{\beta_1 \gamma_1} \cdots g_{\beta_{p-1} \gamma_{p-1}}
		\delta^{\gamma_1}_{[\mu_1} \cdots \delta^{\gamma_{p-1}}_{\mu_{p-1}}
		\nabla_{\beta_p]} X^{\beta_1 \cdots \beta_{p-1} \beta_p} \\
		&= \pm (-1)^{p(n-p)} \nabla^{\alpha}
		X_{[\mu_1 \cdots \mu_{p-1} \alpha]} \\
		&= \pm (-1)^{p (n-p)} \nabla^\alpha X_{\mu_1 \cdots \mu_{p-1} \alpha}
	\end{align}
	ここで
	\begin{align}
		X &= a_{\mu_1 \cdots \mu_p}
		\ddif x^{\mu_1} \wedge \cdots \wedge \ddif x^{\mu_p} \\
		&= p! a_{[\mu_1 \cdots \mu_p]}
		\ddif x^{\mu_1} \otimes \cdots \otimes \ddif x^{\mu_p}
	\end{align}
	となることを用いた。
\end{proof}

これを用いた例として、3次元ユークリッド空間において通常のベクトル解析の作用素は
\begin{align}
	\nabla f = \ddif f \qquad
	\Div X = \star \ddif \star X \qquad
	\Rot X = \star \ddif X
	\tag{8.5} \label{8.5}
\end{align}
と書ける。
\begin{screen}
	\underline{補足}

	3次元ユークリッド空間では、
	デカルト座標を選べば$\epsilon_{ijk} = \varepsilon_{ijk}$で
	\begin{gather}
		\ddif f = (\partial_i f) \ddif x^i \\
		\begin{aligned}
			\star \ddif \star X &= + (-1)^{1 \cdot 2} \nabla^b X_b \\
			&= \partial^i X_i
		\end{aligned} \\
		\begin{aligned}
			\star \ddif X &=
			\star \left[ (\partial_i X_j) \ddif x^i \wedge \ddif x^j\right] \\
			&= \star \left[ (2! \partial_i X_j)
			\ddif x^i \otimes \ddif x^j \right] \\
			&= \left( \frac{1}{2!} \varepsilon_{ijk} \cdot
			2! \partial^j X^k \right) \ddif x^i \\
			&= \left( \varepsilon_{ijk} \partial^j X^k \right) \ddif x^i
		\end{aligned}
	\end{gather}
\end{screen}
ここで$f$は関数、$X$はベクトル場$X^a$から得られる1形式$X_a$を表している。
最後の式は外微分が$\Rot$の一般化と思えることを示している。

他の例として、Maxwell方程式
\begin{align}
	\nabla^a F_{ab} = -4\pi j_b \qquad \nabla_{[a} F_{bc]} = 0
\end{align}
は
\begin{align}
	\ddif \star F = -4 \pi \star j \qquad
	\ddif F = 0
	\tag{8.7} \label{8.7}
\end{align}
と書ける。
\begin{screen}
	\underline{補足}
	\begin{align}
		(\star \ddif \star F)_a = - (-1)^{2 \cdot 2} \nabla^b F_{ab}
		= -\nabla^b F_{ab} = \nabla^b F_{ba}
	\end{align}
	であるから
	\begin{align}
		-(-1)^{3 \cdot 1} (\ddif \star F)_{abc}
		&= (\star \star \ddif \star F)_{abc} \\
		&= \frac{1}{1!} \epsilon_{abcd} (\star \ddif \star F)^d \\
		&= \epsilon_{abcd} \nabla_e F^{ed}
	\end{align}
	となり、
	\begin{align}
		(\ddif \star F)_{abc} = \epsilon_{abcd} \nabla_e F^{ed}
	\end{align}
	が分かる。
	また
	\begin{align}
		(\star j)_{abc} = \frac{1}{1!} \epsilon_{abcd} j^d
		= \epsilon_{abcd} j^d
	\end{align}
\end{screen}

この第1式は$\ddif \star j = 0$つまり$\nabla_a j^a = 0$で、
$j^a$が保存量であることを示している。
第2式はPoincar\'{e}の補題に対応している。

次にspacelikeな超曲面$\Sigma$を考える。
すれば$\Sigma$における電荷の総量を
\begin{align}
	Q = - \int_{\Sigma} * j \tag{8.8} \label{8.8}
\end{align}
と定義できる。
($\Sigma$を$J^-(\Sigma)$の境界とみなし、
Stokesの定理で用いる向きを選ぶことによって$\Sigma$の向きをfixしておく。)
\begin{screen}
	\underline{補足}

	元々の正のフレームが$(v_1, \cdots, v_n)$だったときに、
	$x_1 = {\rm const}$境界の外向きのベクトルが$v$であるとき、
	$(v, v_2, \cdots, v_n)$が正のフレームであるとき
	境界で$(v_2, \cdots, v_n)$を正のフレームとする。
\end{screen}

Maxwell方程式を用いれば
\begin{align}
	Q = \frac{1}{4\pi} \int_{\Sigma} \ddif \star F \tag{8.9} \label{8.9}
\end{align}
で、Stokesの定理を用いれば
\begin{align}
	Q = \frac{1}{4\pi} \int_{\partial \Sigma} \star F \tag{8.10} \label{8.10}
\end{align}
となる。
$\Sigma$上の電荷総量が$\partial \Sigma$上の$\star F$の積分で書ける。
これは曲がった空間におけるGaussの法則$Q \sim \int \vec{E} \cdot \ddif \vec{S}$
の一般化になっている。

例えば体積要素が
$r^2 \sin \theta \ddif t \wedge \ddif r \wedge \ddif \theta \wedge \ddif \phi$
となるように向きを選んだ極座標Minkowski時空で$\Sigma$を$t=0$に選ぶ。
$\Sigma$を$t \le 0$の境界と思ってStokesの定理による向きを考えれば、
$\Sigma$での向きは$\ddif r \wedge \ddif \theta \wedge \ddif \phi$となる。
ここで新たに$\Sigma_R$を$\Sigma$の$r \le R$領域、その境界を$S_R^2$とすれば、
向きは$\ddif \theta \wedge \ddif \phi$となる。
Coulombポテンシャルを考えれば、
\begin{align}
	A = -\frac{q}{r} \ddif t \qquad \Rightarrow
	F = -\frac{q}{r^2} \ddif t \wedge \ddif r
	= -\frac{q}{r^2}
	\left( \ddif t \otimes \ddif r - \ddif r \otimes \ddif t \right)
\end{align}
となって、ホッジ双対は
\begin{align}
	(\star F)_{\mu \nu} &= \epsilon_{\mu \nu \rho \sigma} F^{\rho \sigma} \\
	&= r^2 \sin \theta \cdot \varepsilon_{\mu \nu \rho \sigma}
	g^{\rho \alpha} g^{\sigma \beta} F_{\alpha \beta} \\
	&= \varepsilon_{\mu \nu t r} r^2 \sin \theta
	g^{tt} g^{rr} (F_{tr} - F_{rt}) \\
	&= 2 \varepsilon_{\mu \nu t r} r^2 \sin \theta \cdot \frac{q}{r^2} \\
	&= 2 \epsilon_{\mu \nu t r} q \sin \theta \\
	\star F &= 2 q \sin \theta
	(\ddif \theta \otimes \ddif \phi - \ddif \phi \otimes \ddif \theta) \\
	&= q \sin \theta \ \ddif \theta \wedge \ddif \phi
\end{align}
であるから、
$\Sigma_R$内の総電荷は
\begin{align}
	Q[\Sigma_R]
	= \frac{1}{4\pi} \int_{S_R^2} \ddif \theta \ddif \phi \ q \sin \theta
	= q
\end{align}
となり、欲しい結果が求まった。

Minkowski時空上の漸近的平坦な超曲面で、
$R \to \infty$の極限を取ることで$\Sigma$上の総電荷を無限遠での積分で書くことができる。
これに倣ってあらゆる漸近的平坦な終端(?)での総電荷を定義できる。
\begin{definition}
	$(\Sigma, h_{ab}, K_{ab})$を漸近的平坦な終端とすれば、そこでの総電荷、総磁荷は
	\begin{align}
		Q = \frac{1}{4\pi} \lim_{r \to \infty} \int_{S_r^2} \star F \qquad
		P = \frac{1}{4\pi} \lim_{r \to \infty} \int_{S_r^2} F
	\end{align}
	である。
	ここで$S_2^2$は漸近的平坦な終端の定義で用いられている座標$x^i$において、
	$x^i x^i = r^2$を満たす球面である。
\end{definition}
\begin{exercise}
	この定義がKerr-Newman解での$Q, P$と一致することを示せ。
\end{exercise}
\begin{answer}
	Pass
\end{answer}

従って、時空内にchargeがない(例えば$j^a = 0$)場合でもchargeは有限になりうる。
Kerr-Newman解またはReissner-Nordstrom解において$t$一定面を考えると、
その面でのchargeは0になる。
しかしこれを無限遠での表面積分に置き換えると、
境界には2つの漸近的平坦な終端があるために、2つの項を得る。
したがってこれら2つの終端でのchargeは大きさが等しく逆符号になっていなければならない。

\end{document}