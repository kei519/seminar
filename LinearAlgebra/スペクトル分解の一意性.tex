\documentclass[a4paper, 10pt]{jsarticle}
% 余白
\usepackage[top=20truemm, bottom=25truemm, left=22truemm, right=22truemm, driver=dvipdfm, truedimen, margin=2cm]{geometry}
% 数式
\usepackage{amsmath, amssymb, amsthm}
\usepackage{ascmac}
\usepackage{mathtools}
\usepackage{braket}
\mathtoolsset{showonlyrefs,showmanualtags} 	 % 相互参照した式のみに番号を振る
% 画像
\usepackage[dvipdfmx]{graphicx}
\usepackage[subrefformat=parens]{subcaption}
\captionsetup{compatibility=false}
% ハイパーリンク
\usepackage[dvipdfmx, bookmarksnumbered]{hyperref}
\usepackage{pxjahyper}
\hypersetup{colorlinks=true, linkcolor=black, citecolor=black, urlcolor=black}
% ページを跨ぐ枠
\usepackage{tcolorbox}
\tcbuselibrary{breakable, skins, theorems}

% コマンド定義
\def\vec#1{\mbox{\boldmath $#1$}}
\newcommand{\dif}[2]{\frac{{\rm d} #1}{{\rm d} #2}}
\newcommand{\pdif}[2]{\frac{\partial #1}{\partial #2}}
\newcommand{\ddif}{{\rm d}}
\newcommand{\Ketbra}[2]{\Ket{#1} \! \! \Bra{#2}}
\DeclareMathOperator{\Div}{div}
\DeclareMathOperator{\Grad}{grad}
\DeclareMathOperator{\Rot}{rot}
\renewcommand{\proofname}{証明}
\allowdisplaybreaks[4] 	 % 数式等のページ分割をさせる

% 定理環境
\newcounter{thetcbcounter}
\newtcbtheorem{thm}{定理}{
coltitle = white,
colback = white,
colframe = black!50,
fonttitle = \bfseries,
breakable = true,
}{thm}
\newtcbtheorem[use counter from = thm]{dfn}{定義}{
coltitle = white,
colback = white,
colframe = black!50,
fonttitle = \bfseries,
breakable = true,
}{def}
\newtcbtheorem[use counter from = thm]{lem}{補題}{
coltitle = white,
colback = white,
colframe = black!50,
fonttitle = \bfseries,
breakable = true,
}{lem}
\newtcbtheorem[use counter from = thm]{prop}{命題}{
coltitle = white,
colback = white,
colframe = black!50,
fonttitle = \bfseries,
breakable = true,
}{prop}
\newtcbtheorem[use counter from = thm]{cor}{系}{
coltitle = white,
colback = white,
colframe = black!50,
fonttitle = \bfseries,
breakable = true,
}{cor}
\newtcbtheorem[use counter from = thm]{ass}{仮定}{
coltitle = white,
colback = white,
colframe = black!50,
fonttitle = \bfseries,
breakable = true,
}{ass}
\newtcbtheorem[use counter from = thm]{conj}{予想}{
coltitle = white,
colback = white,
colframe = black!50,
fonttitle = \bfseries,
breakable = true,
}{conj}

\newcommand{\id}{\mathrm{id}}

\title{スペクトル分解の一意性}
\author{}

\begin{document}
\maketitle

まず以下の補題について考えておく。

\begin{lem}{}{完全性}
	各射影子$P_i$($i = 1, 2, \cdots, k$)が
	$P_i P_j = p_i \delta_{ij}$を満たすとき、
	\begin{align}
		\id_V = P_1 + P_2 + \cdots + P_k
		\iff V = P_1(V) \oplus P_2(V) \oplus \cdots \oplus P_k(V)
	\end{align}
\end{lem}
\begin{proof}
	$(\impliedby)$
	条件より、
	任意の$x \in V$について
	$x = \vec{x}_1 + \vec{x}_2 + \cdots + \vec{x}_k$
	(ただし各$i$について$\vec{x}_i \in P_i(V)$)
	とできる。
	ここで各々の$\vec{x}_i$についてある$\vec{y}$が存在して
	$\vec{x}_i = P_i \vec{y}_i$とできるから、
	\begin{align}
		P_i \vec{x}_j = P_i P_j \vec{y}_j
		= P_i \delta_{ij} \vec{y}_j
		= P_i \vec{y}_i \delta_{ij}
		= \vec{x}_i \delta_{ij}
	\end{align}
	が成り立つ。
	したがって
	\begin{align}
		P_i \vec{x}
		&= P_i \left( \vec{x}_1 + \vec{x}_2 + \cdots + \vec{x}_k \right) \\
		&= \vec{x}_i
	\end{align}
	となるから、
	任意の$\vec{x} \in V$について
	\begin{align}
		\vec{x}
		&= \vec{x}_1 + \vec{x}_2 + \cdots + \vec{x}_k \\
		&= \left( P_1 + P_2 + \cdots + P_k \right) \vec{x} \\
		&= \id_V \vec{x}
	\end{align}
	が得られる。

	\noindent $(\implies)$
	まず任意の$x \in V$について
	\begin{align}
		\vec{x}
		&= \id_V \vec{x} \\
		&= \left( P_1 + P_2 + \cdots + P_k \right) \vec{x} \\
		&= \vec{x}_1 + \vec{x}_2 + \cdots + \vec{x}_k
		\qquad \left( \vec{x}_i \coloneqq P_i \vec{x} \in P_i (V) \right)
	\end{align}
	となるから、
	$V = P_1(V) + P_2(V) + \cdots + P_k(V)$
	が分かる。

	次に各$\vec{x}_i \in P_i(V)$について
	$P_i \vec{x}_i = \vec{x}_i$だから、
	\begin{align}
		\left( \vec{x}_i, \vec{x}_j \right)
		&= \left( P_i \vec{x}_i, P_j \vec{x}_j \right) \\
		&= \left( \vec{x}_i, P_i P_j \vec{x}_j \right) \\
		&= 0
	\end{align}
	である。
	したがって
	$\vec{x} \in P_i(V) \cap \left( P_1 (V) + \cdots + P_{i-1}(V) + P_{i+1}(V)
	+ \cdots + P_k(V) \right)$を取ってくると
	\begin{align}
		\left\| \vec{x} \right\|^2
		&= \left( \vec{x}, \vec{x} \right) \\
		&= \left( \vec{x}_i, \vec{x}_1 \right)
		+ \left( \vec{x}_i, \vec{x}_{i-1} \right)
		+ \left( \vec{x}_i, \vec{x}_{i+1} \right)
		+ \left( \vec{x}_i, \vec{x}_k \right) \\
		&= 0
	\end{align}
	となってしまう。
	つまり
	$P_i(V) \cap \left( P_1 (V) + \cdots + P_{i-1}(V) + P_{i+1}(V)
	+ \cdots + P_k(V) \right) = \{ \vec{0} \}$
	が得られる。
	したがって[4.9]より
	\begin{align}
		P_1(V) + P_2(V) + \cdots + P_k(V)
		= V = P_1(V) \oplus P_2(V) \oplus \cdots \oplus P_k(V)
	\end{align}
	である。
\end{proof}

これを踏まえて以下の定理を示す。

\begin{thm}{スペクトル分解の一意性}{一意性}
	ベクトル空間$V$上の正規変換$T$の相違なる固有値を
	$\beta_1, \beta_2, \cdots, \beta_k$
	とすると、
	それらに対応する射影子
	$P_1, P_2, \cdots, P_k$が存在して
	\begin{gather}
		\id_V = P_1 + P_2 + \cdots + P_k \\
		P_i P_j = P_i \delta_{ij}
	\end{gather}
	を満たし、
	$T$を
	\begin{align}
		T = \beta_1 P_1 + \beta_2 P_2 + \cdots + \beta_k P_k
	\end{align}
	と一意的に表せる。
\end{thm}
\begin{proof}
	(存在性)
	固有値$\beta_i$に対応する固有空間を$W_i$とすると、
	$T$は正規変換だから系[2.5]より
	\begin{align}
		V = W_1 \oplus W_2 \oplus \cdots \oplus W_k
	\end{align}
	と書け、
	異なる固有空間同士は直交する。
	そこで$W_i$への射影子を$P_i$とすれば
	$P_i P_j = P_i \delta_{ij}$を満たし、
	補題\ref{lem:完全性}より
	\begin{align}
		\id_V = P_1 + P_2 + \cdots + P_k
	\end{align}
	とできる。
	また任意の$\vec{x}$について
	\begin{align}
		T \vec{x}
		&= T \left( \vec{x}_1 + \vec{x}_2 + \cdots + \vec{x}_k \right)
		\qquad \left( \vec{x}_i \coloneqq P_i \vec{x} \right) \\
		&= \beta_1 \vec{x}_1 + \beta_2 \vec{x}_2 + \cdots
		+ \beta_k \vec{x}_k \\
		&= \left( \beta_1 P_1 + \beta_2 P_2 + \cdots + \beta_k P_k \right)
		\vec{x}
	\end{align}
	が成り立つから
	\begin{align}
		T = \beta_1 P_1 + \beta_2 P_2 + \cdots + \beta_k P_k
	\end{align}
	とできる。

	\noindent (一意性)
	ある部分空間への射影子の組$\{ Q_i \}_{i=1}^l$が
	\begin{gather}
		\id_V = Q_1 + Q_2 + \cdots + Q_l \\
		Q_i Q_j = Q_i \delta_{ij} \\
		T = \alpha_1 Q_1 + \alpha_2 Q_2 + \cdots + \alpha_l Q_l
	\end{gather}
	を満たすとする。
	このとき$i \neq j$ならば$\alpha_i \neq \alpha_j$として良い。
	なぜならば$i \neq j$で$\alpha_i = \alpha_j$であれば
	$Q_i \to Q_i + Q_j$としても
	\begin{align}
		Q_i^*
		&\to \left( Q_i + Q_j \right)^* \\
		&= Q_i + Q_j \\
		Q_i^2
		&\to \left( Q_i + Q_j \right) \left( Q_i + Q_j \right) \\
		&= Q_i^2 + Q_i Q_j + Q_j Q_i + Q_j^2 \\
		&= Q_i + Q_j \\
		Q_i Q_k
		&\to \left( Q_i + Q_j \right) Q_k \\
		&= 0 \qquad \left( \text{ただし} k \neq i, j \right)
	\end{align}
	を満たすから、
	射影演算子を取り替えることで全ての$i \neq j$について
	$\alpha_i \neq \alpha_j$となるようにできるからである。

	はじめに$\alpha_i$は固有値$\beta_i$と一致し
	$k = l$となることを示す。
	そのためにまず$l \leq k$であることを示す。
	射影子の性質から$Q_i \vec{x} = \vec{x}$となる$\vec{x}$が
	少なくとも1つ存在する。
	その$\vec{x}$を持ってくると
	$Q_j \vec{x} = Q_j Q_i \vec{x}
	= Q_i \delta_{ij} \vec{x} = \vec{x} \delta_{ij}$を満たすから
	\begin{align}
		T \vec{x}
		&= \left( \alpha_1 Q_1 + \cdots + \alpha_i Q_i + \cdots
		+ \alpha_k Q_k \right) \vec{x} \\
		&= \alpha_i \vec{x}
	\end{align}
	となる。
	これは$\alpha_i$が$T$の固有値であることを意味するから、
	ある$1 \leq j \leq k$が存在して$\alpha_i = \beta_j$となる。
	$\beta_1, \beta_2, \cdots, \beta_k$はそれぞれ異なるから
	この$j$はただ1つに定まる。
	そこでその$j$を用いて$\mu_i \coloneqq j$とする。
	すると
	\begin{align}
		T = \beta_{\mu_1} Q_1 + \beta_{\mu_2} Q_2 + \cdots + \beta_{\mu_l} Q_l
	\end{align}
	と書けることになる。
	ここで$i \neq j$のとき$\alpha_i \neq \alpha_j$であることから
	$\beta_{\mu_i} \neq \beta_{\mu_j}$
	すなわち$\mu_i \neq \mu_j$でなければならない。
	これは$l > k$であると鳩の巣原理より満たすことができないため
	$l \leq k$でなけらばならない。
	以上を踏まえて、
	簡単のため$Q_i$のラベルを並べ替えて
	\begin{align}
		T = \beta_1 Q_1 + \beta_2 Q_2 + \cdots + \beta_l Q_l
	\end{align}
	となるようにしておく。

	次に$l < k$とすると矛盾することを示す。
	補題\ref{lem:完全性}より任意の$\vec{x}$は
	$Q_i(V)$の元$\vec{x}_i$を用いて
	\begin{align}
		\vec{x} = \vec{x}_1 + \vec{x}_2 + \cdots + \vec{x}_l
	\end{align}
	と一意に表せられる。
	したがって
	\begin{align}
		T \vec{x}
		= \beta_1 \vec{x}_1 + \beta_2 \vec{x}_2 + \cdots + \beta_l \vec{x}_l
	\end{align}
	となるが、
	$\vec{x}$として$T$の固有値$\beta_i$($l+1 \leq i \leq k$)に対応する
	固有ベクトルを取ると
	\begin{align}
		T \vec{x}
		&= \beta_i \vec{x} \\
		&= \beta_i \vec{x}_1 + \beta_i \vec{x}_2 + \cdots + \beta_i \vec{x}_l
	\end{align}
	ともなる。
	ここで$\beta_j \vec{x}_j \in Q_j(V)$かつ$\beta_i \vec{j} \in Q_j(V)$で
	あることと
	$V = Q_1(V) \oplus Q_2(V) \oplus \cdots \oplus Q_l(V)$であることから
	$\beta_j \vec{x}_j = \beta_i \vec{x}_j$すなわち$\beta_i = \beta_j$で
	なければならないが、これは矛盾。
	以上から$l = k$である。

	最後に$P_i(V) = Q_i(V)$であることを示す。
	$\vec{x} \in P_i(V)$とし
	それを$\vec{y}_i \in Q_i(V)$を用いて
	$\vec{x} = \vec{y}_1 + \vec{y}_2 + \cdots + \vec{y}_k$と
	一意的に表しておく。
	すると
	\begin{align}
		T \vec{x}
		&= \left( \beta_1 P_1 + \beta_2 P_2 + \cdots + \beta_k P_k \right)
		\vec{x}
		= \beta_i \vec{x}
		= \beta_i \vec{y}_1 + \beta_i \vec{y}_2 + \cdots + \beta_i \vec{y}_k \\
		&= \left( \beta_1 Q_1 + \beta_2 Q_2 + \cdots + \beta_k Q_k \right)
		\left( \vec{y}_1 + \vec{y}_2 + \cdots + \vec{y}_k \right)
		= \beta_1 \vec{y}_1 + \beta_2 \vec{y}_2 + \cdots + \beta_k \vec{y}_k
	\end{align}
	となる。
	この表し方は一意的だから各$j$について$\beta_i \vec{y}_j = \beta_j \vec{y}_j$
	でなければならないが、
	$j \neq i$については$\vec{y}_j = \vec{0}$でなければ成り立たない。
	つまり$\vec{x} = \vec{y}_i \in Q_i(V)$とならなければならない。
	逆に$\vec{x} \in Q_i(V)$についても同様にして$\vec{x} \in P_i(V)$が言える。
	したがって$P_i(V) = Q_i(V)$となる。
	以上のことから$Q_i = P_i$でなければならない。
\end{proof}

\end{document}