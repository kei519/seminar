\documentclass[a4paper, 10pt]{jsarticle}
% 余白
\usepackage[top=20truemm, bottom=25truemm, left=22truemm, right=22truemm, driver=dvipdfm, truedimen, margin=2cm]{geometry}
% 数式
\usepackage{amsmath, amssymb, amsthm}
\usepackage{ascmac}
\usepackage{mathtools}
\usepackage{braket}
\mathtoolsset{showonlyrefs,showmanualtags} 	 % 相互参照した式のみに番号を振る
% 画像
\usepackage[dvipdfmx]{graphicx}
\usepackage[subrefformat=parens]{subcaption}
\captionsetup{compatibility=false}
% ハイパーリンク
\usepackage[dvipdfmx, bookmarksnumbered]{hyperref}
\usepackage{pxjahyper}
\hypersetup{colorlinks=true, linkcolor=black, citecolor=black, urlcolor=black}
% ページを跨ぐ枠
\usepackage{tcolorbox}
\tcbuselibrary{breakable, skins, theorems}

% コマンド定義
\def\vec#1{\mbox{\boldmath $#1$}}
\newcommand{\dif}[2]{\frac{{\rm d} #1}{{\rm d} #2}}
\newcommand{\pdif}[2]{\frac{\partial #1}{\partial #2}}
\newcommand{\ddif}{{\rm d}}
\newcommand{\Ketbra}[2]{\Ket{#1} \! \! \Bra{#2}}
\DeclareMathOperator{\Div}{div}
\DeclareMathOperator{\Grad}{grad}
\DeclareMathOperator{\Rot}{rot}
\renewcommand{\proofname}{証明}
\allowdisplaybreaks[4] 	 % 数式等のページ分割をさせる

% 定理環境
\newcounter{thetcbcounter}
\newtcbtheorem{thm}{定理}{
coltitle = white,
colback = white,
colframe = black!50,
fonttitle = \bfseries,
breakable = true,
}{thm}
\newtcbtheorem[use counter from = thm]{dfn}{定義}{
coltitle = white,
colback = white,
colframe = black!50,
fonttitle = \bfseries,
breakable = true,
}{def}
\newtcbtheorem[use counter from = thm]{lem}{補題}{
coltitle = white,
colback = white,
colframe = black!50,
fonttitle = \bfseries,
breakable = true,
}{lem}
\newtcbtheorem[use counter from = thm]{prop}{命題}{
coltitle = white,
colback = white,
colframe = black!50,
fonttitle = \bfseries,
breakable = true,
}{prop}
\newtcbtheorem[use counter from = thm]{cor}{系}{
coltitle = white,
colback = white,
colframe = black!50,
fonttitle = \bfseries,
breakable = true,
}{cor}
\newtcbtheorem[use counter from = thm]{ass}{仮定}{
coltitle = white,
colback = white,
colframe = black!50,
fonttitle = \bfseries,
breakable = true,
}{ass}
\newtcbtheorem[use counter from = thm]{conj}{予想}{
coltitle = white,
colback = white,
colframe = black!50,
fonttitle = \bfseries,
breakable = true,
}{conj}

\usepackage{qcircuit}
\DeclareMathOperator*{\Tr}{Tr}

\title{P.168の確率}
\author{}

\begin{document}
\maketitle

以下の回路においての$o$ので観測確率を$p(o)$とする。
\[
\begin{array}{c}
	\Qcircuit @C=1.3em @R=.8em {
		\lstick{} & \multigate{3}{V_x} & \meter & o \\
		\lstick{} & \ghost{V_x} & \qw & \\
		\lstick{} & \ghost{V_x} & \qw & \\
		\lstick{} & \ghost{V_x} & \qw & \
		\inputgroupv{1}{4}{.8em}{.8em}{\Ket{1^n}}
	}
\end{array}
\]
この回路を使って
\[
\begin{array}{c}
	\Qcircuit @C=1.3em @R=.8em {
	\lstick{\Ket{0}} & \targ & \qw & \ctrl{1} & \qw & \targ & \meter & o \\
	\lstick{} & \ctrl{-1} & \multigate{3}{V_x} & \control \qw &
	\multigate{3}{V_x^\dagger} & \ctrl{-1} & \qw & \\
	\lstick{} & \ctrl{-1} & \ghost{V_x} & \qw &
	\ghost{V_x^\dagger} & \ctrl{-1} & \qw & \\
	\lstick{} & \ctrl{-1} & \ghost{V_x} & \qw &
	\ghost{V_x^\dagger} & \ctrl{-1} & \qw & \\
	\lstick{} & \ctrl{-1} & \ghost{V_x} & \qw &
	\ghost{V_x^\dagger} & \ctrl{-1} & \qw & \
	\inputgroupv{2}{5}{.8em}{.8em}{\displaystyle \frac{I^{\otimes n}}{2^n}}
}
\end{array}
\]
という回路を考えたときに$o$での観測確率$\tilde{p}(o = 1)$を求める。

まず状態の遷移がどうなるかを考える。
\begin{align}
	\Ketbra{0}{0} \otimes \frac{I^{\otimes n}}{2^n}
	&\to \frac{1}{2^n} \left[ \Ketbra{0}{0} \otimes
	\underbrace{\left( \cdots \right)}_{\Ketbra{1^n}{1^n}\text{なし}}
	+ \Ketbra{1}{1} \otimes \Ketbra{1^n}{1^n} \right] \\
	&\to \frac{1}{2^n} \left[ \Ketbra{0}{0} \otimes
	V_x \left( \cdots \right) V_x^\dagger
	+ \Ketbra{1}{1} \otimes V_x \Ketbra{1^n}{1^n} V_x^\dagger \right] \\
	&\to \frac{1}{2^n} \left[ \Ketbra{0}{0} \otimes
	V_x \left( \cdots \right) V_x^\dagger
	+ \Ketbra{1}{1} \otimes Z_1 V_x \Ketbra{1^n}{1^n}
	V_x^\dagger Z_1^\dagger \right] \\
	&\to \frac{1}{2^n} \left[ \Ketbra{0}{0} \otimes
	\underbrace{\left( \cdots \right)}_{\Ketbra{1^n}{1^n}\text{なし}}
	+ \Ketbra{1}{1} \otimes V_x^\dagger Z_1 V_x \Ketbra{1^n}{1^n}
	V_x^\dagger Z_1^\dagger V_x \right] \\
	&\to \frac{1}{2^n} \left[ \Ketbra{0}{0} \otimes \left( \cdots \right)
	+ \Ketbra{1}{1} \otimes \left( V_x^\dagger Z_1 V_x \Ketbra{1^n}{1^n}
	V_x^\dagger Z_1^\dagger V_x - A \right)
	+ \Ketbra{0}{0} \otimes A \right]
\end{align}
ここで$A$の部分を計算すると
\begin{align}
	A &= \Tr \left[ V_x^\dagger Z_1 V_x \Ketbra{1^n}{1^n}
	V_x^\dagger Z_1^\dagger V_x \cdot \Ketbra{1^n}{1^n} \right]
	\Ketbra{1^n}{1^n}\\
	&= \left| \Braket{1^n | V_x^\dagger Z_1 V_x | 1^n} \right|^2
	\Ketbra{1^n}{1^n}
\end{align}
である。
絶対値の中身を取り出すと
\begin{align}
	\Braket{1^n | V_x^\dagger Z_1 V_x | 1^n}
	&= \Braket{1^n | V_x^\dagger Z_1 \left\{ \left( \Ketbra{0}{0}
	+ \Ketbra{1}{1} \right) \otimes I^{\otimes n-1} \right\}
	V_x | 1^n} \\
	&= \Braket{1^n | V_x^\dagger \left( \Ketbra{0}{0}
	\otimes I^{\otimes n-1} \right) V_x | 1^n}
	- \Braket{1^n | V_x^\dagger \left( \Ketbra{1}{1}
	\otimes I^{\otimes n-1} \right) V_x | 1^n} \\
	&= \Tr \left[ \left( \Ketbra{0}{0} \otimes I^{\otimes n-1} \right)
	V_x \Ketbra{1^n}{1^n} V_x^\dagger \right]
	- \Tr \left[ \left( \Ketbra{1}{1} \otimes I^{\otimes n-1} \right)
	V_x \Ketbra{1^n}{1^n} V_x^\dagger \right] \\
	&= p(o = 0) - p(o = 1)
\end{align}
を得る。
以上から最終的に
\begin{align}
	p\left( \tilde{o} = 1 \right)
	&= \Tr \left[ \left( \Ketbra{1}{1} \otimes I^{\otimes n} \right)
	\rho \right] \\
	&= \frac{1}{2^n} \Tr \left[ V_x^\dagger Z_1 V_x \Ketbra{1^n}{1^n}
	V_x^\dagger Z_1^\dagger V_x - A \right] \\
	&= \frac{1 - \Tr \left[ A \right]}{2^n} \\
	&= \frac{\left( p(o = 0) + p(o = 1) \right)^2 
	- \left( p(o = 0) - p(o = 1) \right)^2}{2^n} \\
	&= \frac{4 p(o = 0) p(o = 1)}{2^n}
\end{align}
が成り立つことが分かる。

\end{document}