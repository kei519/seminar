\documentclass[a4paper, 10pt]{jsarticle}
% 余白
\usepackage[top=20truemm, bottom=25truemm, left=22truemm, right=22truemm, driver=dvipdfm, truedimen, margin=2cm]{geometry}
% 数式
\usepackage{amsmath, amssymb, amsthm}
\usepackage{ascmac}
\usepackage{mathtools}
\usepackage{braket}
\mathtoolsset{showonlyrefs,showmanualtags} 	 % 相互参照した式のみに番号を振る
% 画像
\usepackage[dvipdfmx]{graphicx}
\usepackage[subrefformat=parens]{subcaption}
\captionsetup{compatibility=false}
% ハイパーリンク
\usepackage[dvipdfmx, bookmarksnumbered]{hyperref}
\usepackage{pxjahyper}
\hypersetup{colorlinks=true, linkcolor=black, citecolor=black, urlcolor=black}
% ページを跨ぐ枠
\usepackage{tcolorbox}
\tcbuselibrary{breakable, skins, theorems}

% コマンド定義
\def\vec#1{\mbox{\boldmath $#1$}}
\newcommand{\dif}[2]{\frac{{\rm d} #1}{{\rm d} #2}}
\newcommand{\pdif}[2]{\frac{\partial #1}{\partial #2}}
\newcommand{\ddif}{{\rm d}}
\newcommand{\Ketbra}[2]{\Ket{#1} \! \! \Bra{#2}}
\DeclareMathOperator{\Div}{div}
\DeclareMathOperator{\Grad}{grad}
\DeclareMathOperator{\Rot}{rot}
\renewcommand{\proofname}{証明}
\allowdisplaybreaks[4] 	 % 数式等のページ分割をさせる

% 定理環境
\newcounter{thetcbcounter}
\newtcbtheorem{thm}{定理}{
coltitle = white,
colback = white,
colframe = black!50,
fonttitle = \bfseries,
breakable = true,
}{thm}
\newtcbtheorem[use counter from = thm]{dfn}{定義}{
coltitle = white,
colback = white,
colframe = black!50,
fonttitle = \bfseries,
breakable = true,
}{def}
\newtcbtheorem[use counter from = thm]{lem}{補題}{
coltitle = white,
colback = white,
colframe = black!50,
fonttitle = \bfseries,
breakable = true,
}{lem}
\newtcbtheorem[use counter from = thm]{prop}{命題}{
coltitle = white,
colback = white,
colframe = black!50,
fonttitle = \bfseries,
breakable = true,
}{prop}
\newtcbtheorem[use counter from = thm]{cor}{系}{
coltitle = white,
colback = white,
colframe = black!50,
fonttitle = \bfseries,
breakable = true,
}{cor}
\newtcbtheorem[use counter from = thm]{ass}{仮定}{
coltitle = white,
colback = white,
colframe = black!50,
fonttitle = \bfseries,
breakable = true,
}{ass}
\newtcbtheorem[use counter from = thm]{conj}{予想}{
coltitle = white,
colback = white,
colframe = black!50,
fonttitle = \bfseries,
breakable = true,
}{conj}

% http://tony.in.coocan.jp/latex/index.html#captype
\makeatletter
\newcommand{\tblcaption}[1]{\def\@captype{table}\caption{#1}}
\makeatother

\title{\S 1 \ 隠れた変数の理論と量子力学}
\author{}

\begin{document}
\maketitle

\section{隠れた変数の理論と量子力学}
\subsection{はじめに}\label{subsec:はじめに}
量子力学の不思議な特徴を、サイコロを用いた喩えで説明する。
まず置かれているサイコロの1番上の面の目が何であるかを測定することを考える。
実際に測定してみたところ、目は1であったとする。
次に今度はその横のうちのどこかの面を測定する。
そうすると5の目が出たとする。
ここで再び1番上の面を測定すると、1ではなく3が出た。
このようなことが起こるのが量子力学である。

古典力学的な見方で考えれば、このようなことは起こらず、
加えてもしサイコロを振ったとしても、
サイコロを作っている粒子や、周りの状態、振り方などを徹底的に調べれば、
ニュートン方程式から予測できるはずである。
このような考え方は\textbf{決定論的}と呼ばれる。

先程のサイコロの例であっても、
実験的にまだ見つかっていない何かしらの変数があって、
それらを踏まえれば出る目が正確に予測できるというように考えれば、
決定論的な理論に従っていると思える。
このような理論は\textbf{隠れた変数の理論}と呼ばれる。
しかしこれは\ref{subsec:実験否定}節で見るように実験的に否定されている。

つまり実際には、いくら詳しく調べたとしても、
どの目が出るかを正確に予言することは不可能である。
このことを原理とした非決定論的な理論が量子力学である。

ここからは、隠れた変数の理論が否定された経緯を、
シュテルン=ゲルラッハ実験と、
量子力学的粒子が持つ固有の角運動量自由度である「スピン」を用いて説明する。

\subsection{シュテルン=ゲルラッハ実験とスピン}\label{subsec:SG実験}
\subsubsection{実験の原理}
図1.4のように不均一な磁場を作る2つの磁石からなるSG装置を考える。
古典力学・電磁気学で考えると、
磁気モーメント$\vec{\mu} = (\mu_x, \mu_y, \mu_z)$を持つ
電気的に中性な物体は、
この磁場から近似で$\mu_z \pdif{B_z}{z}$の力を受ける。

この装置に磁気モーメントを持ち中性な粒子である銀原子を、
$\vec{\mu}$の方向を揃えずに通す実験を考える。

\subsubsection{スピンの上向き状態と下向き状態}
実験結果を予測すると、
$\vec{\mu}$の向きは揃えられていないが、
同じ原子を用いれば$|\vec{\mu}|$は同じはずなので、
上下の幅が同じ1本の線上にビームが分布するであろうと思われる。
しかし、実際の結果は上下2本の細いビームへ分解されただけであった。

磁場は(測定できる程度では)連続した値を取れることが分かっているから、
この結果は$\mu_z$が特定の2つの値しか取れないことを示唆している。
この特異な現象は
\textbf{方向量子化}(quantization of direction, space quantization)と
呼ばれている。
\textbf{量子}(quantum)という言葉は、
磁気モーメントの$z$成分などの物理量に最小単位があり、
その整数倍しか観測されないという現象を表現するために作られた言葉である。
しかし実際には、物理量が離散的な観測値を取るような場合には、
その現象は\textbf{量子化}(quantization)と呼ばれる。

古典的な物体の磁気モーメントは、その自転角運動量に比例している。
そのためこの銀原子の磁気モーメントも、
なにか新しい角運動量の自由度があって、
それに比例しているだろうと解釈できる。
その角運動量を\textbf{スピン角運動量}(spin angular momentum)と呼ぶ。
スピン角運動量は、
軌道角運動量との和が保存することで起こる
アインシュタイン=ドハース効果などで実験的に確認された。
またスピン角運動量を記述する自由度を、
位置自由度と区別して、
スピン自由度または簡単に\textbf{スピン}(spin)と呼ぶ。

\begin{screen}
	\underline{補足}

	普通の角運動量は位置とその時間微分(速度)から計算できる。
	古典力学ではこのように位置とその時間依存性さえ分かっていれば、
	全ての物理量はそこから計算できた。
	しかし、スピン角運動量はそうではないため、
	新たになにか自由度を増やさなければならない。
\end{screen}
銀原子のように、
電子や陽子も2つの状態を持つスピン自由度を有していることが知られている。
このような物理系を\textbf{2準位スピン系}と呼ぶ。
また一般に$N$個の状態を持つスピンの場合は、
$N$準位スピン系と呼ばれる。

\begin{screen}
	\underline{補足}

	$N$準位という言葉は、後に出てくるように通常、
	大きさの違うエネルギーの$N$個の固有状態を意味することが多い。
	しかし、適当なハミルトニアンを用意すれば、
	$N$個の別のエネルギー固有状態で区別できるため、
	$N$準位スピンと呼ばれる。
	また量子情報の分野では、2準位系をqubit、
	$D$準位系をquditと呼ぶこともある。
\end{screen}

では次に、この出てきた2つのビームを
また同じ方向のSG装置に入れるとどうなるかを実験する。
すると、上方向に出てきたビームはまた上方向に、
下方向に出てきたビームはまた下方向に出てきた。
これ以降は何度同じことを繰り返しても同様となる。
したがって、最初のSG装置は、
測った方向に対して全て揃った2つの状態を準備していると解釈できる。
そこで、上に出てきたビーム中の粒子のスピンは上向き、
下に出てきたビーム中の粒子のスピンは下向きの状態であると
定義することとする。

\subsubsection{傾けたSG装置}
次に図1.6のように傾けたSG装置を用いることによって、
$\vec{n} = (n_x, n_y, n_z) = (0, \sin \theta, \cos \theta)$の
方向で上向きスピンのビームを測ってみる。
この方向に引いた空間軸を$z'$軸と呼ぶこととする。
これも古典的には、$z$方向に上向きであったのだから、
$\cos \theta$に比例した大きさでビームが出てくるように思われるが、
実際にはそうはならず、
$\theta$に依らず、一定の距離で上下のビームに分かれる。
これは、磁気モーメントとそれに比例するスピン角運動量の
方向量子化が、
任意の$\vec{n}$方向の測定で起きていることを意味する。

しかし、$\theta$の大きさは測定結果に全く関与しないわけではない。
何度も測定を繰り返した結果得られる、
$z'$軸で上向きに出る確率$p_{+z'}(\theta)$、
$z'$軸で下向きに出る確率$p_{-z'}(\theta)$は
\begin{gather}
	\begin{gathered}
		p_{+z'} = \cos^2 \left( \frac{\theta}{2} \right) \\
		p_{-z'} = \sin^2 \left( \frac{\theta}{2} \right)
	\end{gathered}
	\label{eq:probability}
\end{gather}
となることが分かっている。

\subsection{隠れた変数の理論の実験的な否定}\label{subsec:実験否定}
\subsubsection{隠れた変数の理論は古典力学的}
節\ref{subsec:はじめに}で触れたように、
隠れた変数の理論は、
まだ見えていない変数があって揺らいで見えるが、
全ての物理量が各時刻で決まっている決定論的な理論である。
このように考えると、最初のSG装置で$z$軸方向に揃えられたスピンが、
他の方向については、まだ見えていない変数によって、
不規則にバラついていただけのようにも解釈できる。
実際に、1つのスピンまでならば式\eqref{eq:probability}を
説明する隠れた変数の理論は作れる。
これは付録G.1で紹介している。

\subsubsection{隠れた変数の理論は正しいのか?}
\label{subsubsec:question}
1個のスピンのSG実験を説明できるならば、
隠れた変数の理論でも悪くないという感触を持つかもしれないが、
ごく自然な条件(あとで説明する)を満たす隠れた変数の理論は
全て実験で否定されている。
以下では、
このことを有名なベルの不等式の簡略版であるCHSH不等式を用いて説明する。

\subsubsection{CHSH不等式の観測量}
2準位スピンを持った2つの粒子が空間的に離れた場所にあるとする。
ここで空間的に離れた2粒子という設定は、
情報の伝達速度は光速を超えられないという、
相対論的な因果律を実験に課す(\textcolor{red}{実験に課すとは?})
ときに使われる。
そしてそれぞれのスピン自由度を、スピン$A$、スピン$B$と呼ぶ。
それぞれの粒子をそれぞれの場所に置かれたSG装置に入れると、
そのスピンの向きに応じて装置の上方または下方から出てくる。
ここで解析のために上方から出てきたスピンには$+1$、
下方から出てきたスピンには$-1$という数字を割り当てる。
ここでのスピンの値は、
$\vec{n}$の方向に依らず、上下に同じ幅で分かれるという、
方向量子化の実験事実から、
$\vec{n}$の向きに依らない絶対値が同じ正負の値に取るのが自然である。

$z$軸に向けられたSG装置で測られるスピン$A$を$\sigma_{zA}$、
$x$軸を中心に$z$軸から$-90^\circ$回転させた方向、
つまり$y$軸方向に向けられたSG装置で測ったスピンAの値を
$\sigma_{yA}$と表記する。
また$x$軸を中心に$z$軸から$+45^\circ$回転させた方向を$z'$軸として、
$z'$軸方向に向けられたSG装置で測ったスピン$B$の値を$\sigma_{z'B}$、
$x$軸を中心に$z$軸から$-45^\circ$回転させた方向を$y'$軸として、
$y'$軸方向に向けられたSG装置で測った$\sigma_{z'B}$と表記する。

このとき
\begin{align}
	D = \sigma_{yA} \left( \sigma_{y'B} - \sigma_{z'B} \right)
	+ \sigma_{zA} \left( \sigma_{y'B} + \sigma_{z'B} \right)
	\label{eq:cor}
\end{align}
という量を考察してみる。
\begin{problem}
	\eqref{eq:cor}式の$D$について、
	隠れた変数の理論では$D = \pm 2$であることを確かめよ。

	\tcblower

	隠れた変数の理論においてそれぞれの$\sigma$は、
	各時刻で$\pm 1$の確定した値を持っているため、
	$\sigma_{y'B} - \sigma_{z' B}$、$\sigma_{y'B} + \sigma_{z' B}$は
	それぞれ

	\begin{center}
		\begin{tabular}{cccc}
			\hline
			$\sigma_{y'B}$ & $\sigma_{z'B}$ &
			$\sigma_{y'B} - \sigma_{z' B}$ &
			$\sigma_{y'B} + \sigma_{z' B}$ \\
			\hline \hline
			$+1$ & $+1$ & $0$ & $+2$ \\
			$+1$ & $-1$ & $+2$ & $0$ \\
			$-1$ & $+1$ & $-2$ & $0$ \\
			$-1$ & $-1$ & $0$ & $-2$ \\
			\hline
		\end{tabular}
		\tblcaption{$\sigma_{y'B} - \sigma_{z' B}$と
		$\sigma_{y'B} - \sigma_{z' B}$の関係}
		\label{tbl:problem}
	\end{center}
	となる。
	ここで$\sigma_{yA}$、$\sigma_{zA}$も$\pm 1$の値しか
	取れないことを思い出すと、
	結局$D = \pm 2$となることが分かる。
\end{problem}
このように隠れた変数の理論において$D = \pm 2$となる。

また$D$を展開すると
\begin{align}
	D = \sigma_{yA} \sigma_{y'B} - \sigma_{yA} \sigma_{z'B}
	+ \sigma_{zA} \sigma_{y'B} + \sigma_{zA} \sigma_{z'B}
	\label{eq:cor_expanded}
\end{align}
とも書ける。
ここでこの式\eqref{eq:cor_expanded}の各項についてもう少し考えてみる。
一般に物理系$A$の物理量$O_A$と物理系$B$の物理量$O_B$を測定し、
その積$O_A O_B$のような値を計算すると、
それは2つの物理系の相互の関係性が読み取れるものとなる。
例えば極端な例として、必ず$O_A O_B = 1$という状況を考えれば、
$O_A$、$O_B$の一方を計測した時点で他方を確実に言い当てることができる。
したがって$O_A O_B$というような量を
相関量(correlation quantity)と呼ぶことにする。
すれば、式\eqref{eq:cor_expanded}は4つの相関量の和となっているため、
先程の極端な例を踏まえると、
$D$の絶対値が大きければ一方のスピンの測定結果から
容易に他方のスピンを予測する事ができると考えられる。
\begin{screen}
	\underline{補足}

	隠れた変数の理論、量子力学では達成できないが、
	$\sigma_{yA} \sigma_{y'B} = 1$、
	$\sigma_{yA} \sigma_{z'B} = -1$、
	$\sigma_{zA} \sigma_{y'B} = 1$、
	$\sigma_{zA} \sigma_{z'B} = 1$
	の場合$D = 4$となり、
	このとき$A$のスピン
	$\sigma_{yA}$または$\sigma_{zA}$が測定できれば、
	$B$のスピン
	$\sigma_{y'B}$または$\sigma_{z'B}$が正確に予言できる。
	このような例は第15章で紹介するPR箱理論で許される。
\end{screen}

次に新たに言葉の定義をしておく。
一般的な物理量に対して、
その値が$a_n$、そして$a_n$が観測される確率が$p_n$である場合、
実験前に期待される$A$の量として
\begin{align}
	\Braket{A} = \sum_n a_n p_n
\end{align}
を期待値(expectation value)と呼ぶ。
また観測される値$a$が連続である場合は、
確率密度$p(a)$を用いて
\begin{align}
	\Braket{A} = \int a p(a) \ddif a
\end{align}
と定義する。

また一般の物理系$A$、$B$、物理量$O_A$、$O_B$の場合に戻ると、
相関量$O_A O_B$についても期待値$\Braket{O_A O_B}$が考えられるが、
$O_A$、$O_B$がそれぞれ独立に分布する場合、
つまり
\begin{align}
	\Pr \left[ O_A = a, O_B = b \right]
	= \Pr \left[ O_A = a \right] \Pr \left[ O_B = b \right]
\end{align}
である場合には、
$O_A = a$であるという事象と、
$O_B = b$であるという事象にはなんの相関もないと考えられる。
しかしこの場合
\begin{align}
	\Braket{O_A O_B} 
	&= \sum_a \sum_b a b \Pr \left[ O_A = a, O_B = b \right] \\
	&= \sum_a a \Pr \left[ O_A = a \right]
	\sum_b b \Pr \left[ O_B = b \right] \\
	&= \Braket{O_A} \Braket{O_B}
\end{align}
となり、
$O_A O_B$にはなんの相関もないのに意味のありそうな
$\Braket{O_A O_B}$になってしまう可能性がある。
(極端には$\Braket{O_A} = 1$、$\Braket{O_B} = 1$のような場合。)
したがってしばしば
\begin{align}
	C_{AB} = \Braket{O_A O_B} - \Braket{O_A} \Braket{O_B}
\end{align}
と定義される共分散(covariance)が用いられる。
またそれを
$\sqrt{\Braket{( O_A - \Braket{O_A})^2}}
\sqrt{\Braket{( O_B - \Braket{O_B})^2}}$
で割ったものを\textbf{相関係数}(correlation coefficient)と呼ぶ。
以後相関係数が非零ならば、
物理量$O_A$と$O_B$との間には
\textbf{相関}(correlation)があると表現する。
相関係数の絶対値が大きいときには、
$O_A$と$O_B$の相関は大きい、または強いと表現する。

\subsubsection{CHSH不等式}
\ref{subsubsec:question}節で述べた、
自然な条件を満たす隠れた変数の理論とは、
同時確率分布
$\Pr \left[ \sigma_{yA}, \sigma_{zA},
\sigma_{y'B}, \sigma_{z'B} \right]$
が存在する理論のことである。
なぜならば、隠れた変数の理論では、
未だ観測されていない変数$\vec{V}$(数が分からないためベクトル)が
存在して、
それも考慮に入れると各時刻での
すべての値を正確に予言できるというものである。
したがって、
$\sigma_{yA}, \sigma_{zA},\sigma_{y'B}, \sigma_{z'B}$が
特定の値を取る場合には、
ある$\vec{v}$が存在して
\begin{align}
	\Pr \left[ \vec{V} = \vec{v} \mid
	\sigma_{yA}, \sigma_{zA}, \sigma_{y'B}, \sigma_{z'B} \right]
	= \frac{\Pr \left[ \vec{V} = \vec{v},
	\sigma_{yA}, \sigma_{zA}, \sigma_{y'B}, \sigma_{z'B} \right]}
	{\Pr \left[ \vec{V} = \vec{v} \right]}
	= 1
	\label{eq:全確率}
\end{align}
となるはずである。
ここで$\Pr \left[ \vec{V} = \vec{v} \right]$は
その$\vec{v}$が実現される初期値の数から、
普通の確率と同様に求めることができるはずである。
したがって条件を満たす$\vec{v}$の集合$U$
全てについて和を取ることで
\begin{align}
	\Pr \left[ \sigma_{yA}, \sigma_{zA},
	\sigma_{y'B}, \sigma_{z'B} \right]
	= \sum_{\vec{v} \in U} \Pr \left[ \vec{V} = \vec{v} \right]
\end{align}
と同時確率分布
$\Pr \left[ \sigma_{yA}, \sigma_{zA},
\sigma_{y'B}, \sigma_{z'B} \right]$
を求めることができる。
それも全ての変数は各時刻において確定した値を持っている。
そういう意味を表すために「同時」という言葉を付けている。

そうすれば$s = y,z$および$s' = y', z'$
という値を取る変数$s, s'$を考えたときに、
\begin{align}
	\Braket{\sigma_{sA} \sigma_{s'B}}
	= \sum_{\sigma_{yA} = \pm 1} \sum_{\sigma_{zA} = \pm 1}
	\sum_{\sigma_{y'B} = \pm 1} \sum_{\sigma_{z'B} = \pm 1}
	\sigma_{yA} \sigma_{zA} \sigma_{y'B} \sigma_{z'B}
	\Pr \left[ \sigma_{yA}, \sigma_{zA},
	\sigma_{y'B}, \sigma_{z'B} \right]
\end{align}
と計算できる。
(\textcolor{red}
{こちらが自然な隠れた変数の理論の定義になっているのはなぜ?})
この場合
$\Braket{D} = \Braket{\sigma_{yA} \sigma_{y'B}}
- \Braket{\sigma_{yA} \sigma_{z'B}}
+ \Braket{\sigma_{zA} \sigma_{y'B}}
+ \Braket{\sigma_{zA} \sigma_{z'B}}$
となり、
また$D = \pm 2$の値しか取れないため、$-2 \leq D \leq 2$となる。
つまり、隠れた変数の理論では
\begin{align}
	-2 \leq
	\Braket{\sigma_{yA} \sigma_{y'B}}
	- \Braket{\sigma_{yA} \sigma_{z'B}}
	+ \Braket{\sigma_{zA} \sigma_{y'B}}
	+ \Braket{\sigma_{zA} \sigma_{z'B}}
	\leq 2
	\label{eq:CHSH}
\end{align}
が成り立つ。
これがCHSH不等式である。
式\eqref{eq:CHSH}中の期待値は全て
2箇所にあるSG装置の向きを変えて行う
4つの独立な実験によって求められる。

\subsubsection{CHSH不等式からチレルソン不等式へ}
式\eqref{eq:CHSH}の各項を実験において確認したところ、
式\eqref{eq:CHSH}を破るようなスピン$A$、$B$の初期状態を
用意できてしまった。
したがって自然な隠れた変数の理論は全て否定されることになる。
この状態は第5章で見る量子もつれ(エンタングル)状態に当たる。
ここで強調すべきこととして、
CHSH不等式は2つの2準位スピン系に限らず、
任意の2つの系を用意して、
その各々の系でそれぞれ特定の2つの状態を指定し、
それらをスピン上向き状態、下向き状態とみなし、
他の状態が出る確率を0にしても成り立つということがある。
つまり、自然な隠れた変数の理論は、
CHSH不等式を破る量子もつれ状態が作れる他のすべての系でも否定
されていると主張できる。

これまでの実験で調べられた範囲では、
式\eqref{eq:CHSH}ではなく
\begin{align}
	-2\sqrt{2} \leq
	\Braket{\sigma_{yA} \sigma_{y'B}}
- \Braket{\sigma_{yA} \sigma_{z'B}}
+ \Braket{\sigma_{zA} \sigma_{y'B}}
+ \Braket{\sigma_{zA} \sigma_{z'B}}
\leq 2\sqrt{2}
\label{eq:チレルソン}
\end{align}
という不等式が成り立っている。
これはチレルソンによって理論的に導かれた量子力学の予言と
厳密に一致し、
\eqref{eq:チレルソン}は\textbf{チレルソン限界}、
\textbf{チレルソン不等式}と呼ばれている。
よって、量子力学は隠れた変数の理論とは違い、
これまでの実験を高い精度で説明する理論になっている。

\vspace{2zw}

\begin{center}
	\textcolor{blue}{\HUGE{以下教科書P.13以降参照。}}
\end{center}

\end{document}