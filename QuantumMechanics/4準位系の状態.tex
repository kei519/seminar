\documentclass[a4paper, 10pt]{jsarticle}
% 余白
\usepackage[top=20truemm, bottom=25truemm, left=22truemm, right=22truemm, driver=dvipdfm, truedimen, margin=2cm]{geometry}
% 数式
\usepackage{amsmath, amssymb, amsthm}
\usepackage{ascmac}
\usepackage{mathtools}
\usepackage{braket}
\mathtoolsset{showonlyrefs,showmanualtags} 	 % 相互参照した式のみに番号を振る
% 画像
\usepackage[dvipdfmx]{graphicx}
\usepackage[subrefformat=parens]{subcaption}
\captionsetup{compatibility=false}
% ハイパーリンク
\usepackage[dvipdfmx, bookmarksnumbered]{hyperref}
\usepackage{pxjahyper}
\hypersetup{colorlinks=true, linkcolor=black, citecolor=black, urlcolor=black}
% ページを跨ぐ枠
\usepackage{tcolorbox}
\tcbuselibrary{breakable, skins, theorems}

% コマンド定義
\def\vec#1{\mbox{\boldmath $#1$}}
\newcommand{\dif}[2]{\frac{{\rm d} #1}{{\rm d} #2}}
\newcommand{\pdif}[2]{\frac{\partial #1}{\partial #2}}
\newcommand{\ddif}{{\rm d}}
\newcommand{\Ketbra}[2]{\Ket{#1} \! \! \Bra{#2}}
\DeclareMathOperator{\Div}{div}
\DeclareMathOperator{\Grad}{grad}
\DeclareMathOperator{\Rot}{rot}
\renewcommand{\proofname}{証明}
\allowdisplaybreaks[4] 	 % 数式等のページ分割をさせる

% 定理環境
\newtcbtheorem{thm}{定理}{
coltitle = white,
colback = white,
colframe = black!50,
fonttitle = \bfseries,
breakable = true,
}{thm}
\newtcbtheorem[use counter from = thm]{dfn}{定義}{
coltitle = white,
colback = white,
colframe = black!50,
fonttitle = \bfseries,
breakable = true,
}{def}
\newtcbtheorem[use counter from = thm]{lem}{補題}{
coltitle = white,
colback = white,
colframe = black!50,
fonttitle = \bfseries,
breakable = true,
}{lem}
\newtcbtheorem[use counter from = thm]{prop}{命題}{
coltitle = white,
colback = white,
colframe = black!50,
fonttitle = \bfseries,
breakable = true,
}{prop}
\newtcbtheorem[use counter from = thm]{cor}{系}{
coltitle = white,
colback = white,
colframe = black!50,
fonttitle = \bfseries,
breakable = true,
}{cor}
\newtcbtheorem[use counter from = thm]{ass}{仮定}{
coltitle = white,
colback = white,
colframe = black!50,
fonttitle = \bfseries,
breakable = true,
}{ass}
\newtcbtheorem[use counter from = thm]{conj}{予想}{
coltitle = white,
colback = white,
colframe = black!50,
fonttitle = \bfseries,
breakable = true,
}{conj}

% https://marukunalufd0123.hatenablog.com/entry/2019/03/15/071717
\newcounter{problemNum}
\newtcolorbox{problem}[1][]{enhanced,
	breakable,
	boxrule=0.5mm,
	top=2pt,left=44pt,right=4pt,bottom=2pt,arc=0mm,
	colframe=blue!30!gray,
	boxrule=1pt,
	#1,
	underlay unbroken and first={
	\node[inner sep=1pt,blue!50!black,fill=blue!10!white]at ([xshift=22pt,yshift=-9pt]interior.north west) {\stepcounter{problemNum}\bfseries\gtfamily 問題\theproblemNum};},
	segmentation code={
	\draw[dashed] (segmentation.west)--(segmentation.east);
	\node[inner sep=1pt,blue!50!black,fill=blue!10!white] at ([xshift=22pt,yshift=-8pt]segmentation.south west) {\bfseries\gtfamily 解};},
	skin first is subskin of={enhancedfirst}{segmentation code={
	\draw[dashed] (segmentation.west)--(segmentation.east);
	\node[inner sep=1pt,blue!50!black,fill=blue!10!white] at ([xshift=22pt,yshift=-8pt]segmentation.south west) {\bfseries\gtfamily 解};}},
	before upper={\setlength{\parindent}{1zw}},
	before lower={\setlength{\parindent}{1zw}},
}

\DeclareMathOperator*{\Tr}{Tr}

\renewcommand{\Re}{\operatorname{Re}}
\renewcommand{\Im}{\operatorname{Im}}

\title{4準位系の状態について}
\author{}

\begin{document}
\maketitle

4準位系について、
2準位系スピン2つの実験からの考察が教科書5.1節にあったが、
$N$準位系の一般論を適応しても同じことが言えるか考えてみる。
ここで$\hat{\sigma}_0 = \hat{I}$、$\hat{\sigma}_1 = \hat{\sigma}_x$、
$\hat{\sigma}_2 = \hat{\sigma}_y$、$\hat{\sigma}_3 = \hat{\sigma}_z$
としておく。

まず2次元エルミート行列の基底は$\{\hat{\sigma}_a\}_{a=0}^3$が張ることから、
4次元エルミート行列、
つまり2次元エルミート行列同士のテンソル積の基底は
$\{ \hat{\sigma}_a \otimes \hat{\sigma}_b \}_{a,b = 0}^3$で与えられる。
($N$次元エルミート行列は実$N^2$次元ベクトル空間であることからも、
これは基底であることが分かる。)
そこで4準位系の密度行列は
\begin{align}
	\hat{\rho} = \frac{1}{4}
	\sum_{a,b = 0}^4 \alpha_{ab} \hat{\sigma}_a \otimes \hat{\sigma}_b
\end{align}
で与えられる。

次に密度演算子の性質から係数$\alpha_{ab}$の条件を求めていく。
まず$\Tr \left[ \hat{\rho} \right] = 1$であるから、
\begin{align}
	\Tr \left[ \hat{\rho} \right]
	= \frac{1}{4} \alpha_{00} \Tr \left[ \hat{I} \otimes \hat{I} \right]
	= \alpha_{00}
	= 1
\end{align}
が得られる。
また
\begin{align}
	\Tr \left[ \hat{\rho}^2 \right]
	&= \frac{1}{16} \sum_{a,a',b,b'} \alpha_{ab} \alpha_{a'b'}
	\Tr \left[ \hat{\sigma}_a \hat{\sigma}_{a'} \right]
	\Tr \left[ \hat{\sigma}_b \hat{\sigma}_{b'} \right] \\
	&= \frac{1}{4} \sum_{a,b = 0}^3 \alpha_{ab}^2 \\
	&\leq 1
\end{align}
から
\begin{align}
	\sum_{a \neq 0 \lor b \neq 0} \alpha^2_{ab} \leq 3
\end{align}
が得られる。
最後に非負性であるが、
ここで単位ベクトルを考えても一般性を失わないから、
\begin{gather}
	\Ket{\Psi} = \mu_{++} \Ket{+} \Ket{+} + \mu_{+-} \Ket{+} \Ket{-}
	+ \mu_{-+} \Ket{-} \Ket{+} + \mu_{--} \Ket{-} \Ket{-} \\
	\left| \mu_{++} \right|^2 + \left| \mu_{+-} \right|^2
	+ \left| \mu_{-+} \right|^2 + \left| \mu_{--} \right|^2 = 1
\end{gather}
に対して
\begin{align}
	\Braket{\Psi | \hat{\rho} | \Psi} \geq 0
\end{align}
という条件を考える。
これを頑張って計算すると
\begin{align}
	\frac{1}{4} &\left[ 1
	+ \left| \mu_{++} \right|^2 \left( \alpha_{03} + \alpha_{30} + \alpha_{33}
	\right)
	+ \left| \mu_{+-} \right|^2 \left( \alpha_{30} - \alpha_{03} - \alpha_{33}
	\right)
	+ \left| \mu_{-+} \right|^2 \left( \alpha_{03} - \alpha_{30} - \alpha_{33}
	\right)
	+ \left| \mu_{--} \right|^2 \left( \alpha_{33} - \alpha_{03} - \alpha_{30}
	\right) \right. \\
	&\quad + 2 \Re \left( \mu_{++}^* \mu_{+-} \right)
	\left( \alpha_{01} + \alpha_{31} \right)
	+ 2 \Im \left( \mu_{++}^* \mu_{+-} \right)
	\left( \alpha_{02} + \alpha_{32} \right) \\
	&\quad + 2 \Re \left( \mu_{++}^* \mu_{-+} \right)
	\left( \alpha_{10} + \alpha_{13} \right)
	+ 2 \Im \left( \mu_{++}^* \mu_{-+} \right)
	\left( \alpha_{20} + \alpha_{23} \right) \\
	&\quad + 2 \Re \left( \mu_{++}^* \mu_{--} \right)
	\left( \alpha_{11} - \alpha_{22} \right)
	+ 2 \Im \left( \mu_{++}^* \mu_{--} \right)
	\left( \alpha_{12} + \alpha_{21} \right) \\
	&\quad + 2 \Re \left( \mu_{+-}^* \mu_{-+} \right)
	\left( \alpha_{11} + \alpha_{22} \right)
	+ 2 \Im \left( \mu_{+-}^* \mu_{-+} \right)
	\left( \alpha_{21} - \alpha_{12} \right) \\
	&\quad + 2 \Re \left( \mu_{+-}^* \mu_{--} \right)
	\left( \alpha_{10} - \alpha_{13} \right)
	+ 2 \Im \left( \mu_{+-}^* \mu_{--} \right)
	\left( \alpha_{20} - \alpha_{23} \right) \\
	&\quad \left. + 2 \Re \left( \mu_{-+}^* \mu_{--} \right)
	\left( \alpha_{01} - \alpha_{31} \right)
	+ 2 \Im \left( \mu_{-+}^* \mu_{--} \right)
	\left( \alpha_{02} - \alpha_{32} \right)
	\right]
	\geq 0 \label{eq:condition}
\end{align}
を得る。
ここで
\begin{align}
	\begin{cases}
		\left| \mu_{++} \right|^2 = 1 \text{とすると} &
		1 + \alpha_{03} + \alpha_{30} + \alpha_{33} \geq 0 \\
		\left| \mu_{+-} \right|^2 = 1 \text{とすると} &
		1 + \alpha_{30} - \alpha_{03} - \alpha_{33} \geq 0 \\
		\left| \mu_{-+} \right|^2 = 1 \text{とすると} &
		1 + \alpha_{03} - \alpha_{30} - \alpha_{33} \geq 0 \\
		\left| \mu_{--} \right|^2 = 1 \text{とすると} &
		1 + \alpha_{33} - \alpha_{03} - \alpha_{30} \geq 0
	\end{cases}
\end{align}
これらを用いると
\begin{align}
	-1 \leq \alpha_{03}, \alpha_{30}, \alpha_{33} \leq 1
\end{align}
が得られる。
また
\begin{align}
	\begin{cases}
		\displaystyle \Ket{\Psi} = \frac{1}{\sqrt{2}}
		\left( \Ket{+}\Ket{+} + \Ket{+} \Ket{-} \right) \text{とすると} &
		1 + \alpha_{30} + \alpha_{01} + \alpha_{31} \geq 0 \\
		\displaystyle \Ket{\Psi} = \frac{1}{\sqrt{2}}
		\left( \Ket{+}\Ket{+} - \Ket{+} \Ket{-} \right) \text{とすると} &
		1 + \alpha_{30} - \alpha_{01} - \alpha_{31} \geq 0 \\
		\displaystyle \Ket{\Psi} = \frac{1}{\sqrt{2}}
		\left( \Ket{-}\Ket{+} + \Ket{-} \Ket{-} \right) \text{とすると} &
		1 - \alpha_{30} + \alpha_{01} - \alpha_{31} \geq 0 \\
		\displaystyle \Ket{\Psi} = \frac{1}{\sqrt{2}}
		\left( \Ket{-}\Ket{+} - \Ket{-} \Ket{-} \right) \text{とすると} &
		1 - \alpha_{30} - \alpha_{01} + \alpha_{31} \geq 0
	\end{cases}
\end{align}
から
\begin{align}
	-1 \leq \alpha_{01}, \alpha_{31} \leq 1
\end{align}
が得られる。
同様にして
\begin{align}
	-1 \leq \alpha_{ab} \leq 1
\end{align}
が得られる(と思う)。

また変数同士の関係も式\eqref{eq:condition}から制限される。
その具体例をチレルソン不等式の上限を満たす場合で考えてみる。
その前にパウリ演算子をブラケットで挟んだ場合に吐く数字を考える。
\begin{align}
	\hat{\sigma}_0 &= \Ketbra{+}{+} + \Ketbra{-}{-} \\
	\hat{\sigma}_1 &= \Ketbra{+}{-} + \Ketbra{-}{+} \\
	\hat{\sigma}_2 &= -i \Ketbra{+}{-} + i \Ketbra{-}{+} \\
	\hat{\sigma}_3 &= \Ketbra{+}{+} - \Ketbra{-}{-}
\end{align}
であるから、
\begin{align}
	\Braket{+ | \hat{\sigma}_a | +} &= \delta_{a0} + \delta_{a3} \\
	\Braket{+ | \hat{\sigma}_a | -} &= \delta_{a1} -i \delta_{a2} \\
	\Braket{- | \hat{\sigma}_a | +} &= \delta_{a1} + i \delta_{a2} \\
	\Braket{- | \hat{\sigma}_a | -} &= \delta_{a0} - \delta_{a3}
\end{align}
が分かる。
したがって
\begin{align}
	\hat{D} = \sqrt{2} \hat{\sigma}_y \otimes \hat{\sigma}_y
	+ \sqrt{2} \hat{\sigma}_z \otimes \hat{\sigma}_z
\end{align}
について、
その上限を達成する状態つまり
\begin{align}
	\Tr \left[ \hat{\rho} \left( \hat{\sigma}_y \hat{\sigma}_y \right) \right]
	= 1
\end{align}
を満たすとき
\begin{align}
	&\quad \Tr \left[ \hat{\rho} \left( -\Ketbra{+}{-} \otimes \Ketbra{+}{-}
	+ \Ketbra{+}{-} \otimes \Ketbra{-}{+}
	+ \Ketbra{-}{+} \otimes \Ketbra{+}{-}
	- \Ketbra{-}{+} \otimes \Ketbra{-}{+} \right) \right] \\
	&=\sum_{a,b} \frac{\alpha_{ab}}{4} \cdot 4\delta_{a2} \delta_{b2} \\
	&= \alpha_{22}
	= 1
\end{align}
でなければならない。
同様に
\begin{align}
	\Tr \left[ \hat{\rho}
	\left( \hat{\sigma}_z \otimes \hat{\sigma}_z \right) \right]
	= \alpha_{33}
	= 1
\end{align}
が得られる。
ここで式\eqref{eq:condition}に戻って
$\displaystyle \Ket{\Psi} = \frac{1}{\sqrt{2}} \left( \Ket{+} \Ket{-}
- \Ket{-}\Ket{+} \right)$を代入すると、
\begin{align}
	1 - \alpha_{11} - \alpha_{22} - \alpha_{33}
	= -1 - \alpha_{11} \geq 0
\end{align}
でなければならないことが分かるから、
結局$\alpha_{11} = -1$でなければならない。
すると
\begin{align}
	\sum_{a,b} \alpha_{ab}^2 \leq 4
	\label{純粋度}
\end{align}
の条件から、
$a \neq b$のとき$\alpha_{ab} = 0$である。
以上をまとめると、
\begin{align}
	\hat{\rho} = \frac{1}{4} \left( \hat{I} \otimes \hat{I}
	- \hat{\sigma}_{x} \otimes \hat{\sigma}_x
	+ \hat{\sigma}_y \otimes \hat{\sigma}_y
	+ \hat{\sigma}_z \otimes \hat{\sigma}_z \right)
\end{align}
がチレルソン不等式の上限、
つまり
\begin{align}
	\Braket{D} = \Tr \left[ \hat{D} \hat{\rho} \right]
	= 2\sqrt{2}
\end{align}
を満たす状態である。
これは式\eqref{純粋度}の等号を成立させる条件だから純粋状態である。
実際ブラケットを用いて書くと
\begin{align}
	\hat{\rho}
	&= \frac{1}{4} \left[ \left( \Ketbra{+}{+} + \Ketbra{-}{-} \right)
	\otimes \left( \Ketbra{+}{+} + \Ketbra{-}{-} \right)
	- \left( \Ketbra{+}{-} + \Ketbra{-}{+} \right) \otimes
	\left( \Ketbra{+}{-} + \Ketbra{-}{+} \right) \right. \\
	&\qquad \qquad \left.
	+ \left( -i \Ketbra{+}{-} +i \Ketbra{-}{+} \right) \otimes
	\left( -i \Ketbra{+}{-} +i \Ketbra{-}{+} \right)
	+ \left( \Ketbra{+}{+} - \Ketbra{-}{-} \right) \otimes
	\left( \Ketbra{+}{+} - \Ketbra{-}{-} \right) \right] \\
	&= \frac{1}{2} \left[ \Ketbra{+}{+} \otimes \Ketbra{+}{+}
	- \Ketbra{+}{-} \otimes \Ketbra{+}{-}
	- \Ketbra{-}{+} \otimes \Ketbra{-}{+}
	+ \Ketbra{-}{-} \otimes \Ketbra{-}{-} \right] \\
	&= \frac{1}{\sqrt{2}} \left( \Ket{+} \Ket{+} - \Ket{-} \Ket{-} \right)
	\cdot \frac{1}{\sqrt{2}} \left( \Bra{+} \Bra{+} - \Bra{-} \Bra{-} \right)
	\\
	&= \Ketbra{\Psi_-}{\Psi_-}
\end{align}
であることが分かる。

\end{document}