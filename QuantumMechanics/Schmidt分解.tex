\documentclass[a4paper, 10pt]{jsarticle}
% 余白
\usepackage[top=20truemm, bottom=25truemm, left=22truemm, right=22truemm, driver=dvipdfm, truedimen, margin=2cm]{geometry}
% 数式
\usepackage{amsmath, amssymb, amsthm}
\usepackage{ascmac}
\usepackage{mathtools}
\usepackage{braket}
\mathtoolsset{showonlyrefs,showmanualtags} 	 % 相互参照した式のみに番号を振る
% 画像
\usepackage[dvipdfmx]{graphicx}
\usepackage[subrefformat=parens]{subcaption}
\captionsetup{compatibility=false}
% ハイパーリンク
\usepackage[dvipdfmx, bookmarksnumbered]{hyperref}
\usepackage{pxjahyper}
\hypersetup{colorlinks=true, linkcolor=black, citecolor=black, urlcolor=black}
% ページを跨ぐ枠
\usepackage{tcolorbox}
\tcbuselibrary{breakable, skins, theorems}

% コマンド定義
\def\vec#1{\mbox{\boldmath $#1$}}
\newcommand{\dif}[2]{\frac{{\rm d} #1}{{\rm d} #2}}
\newcommand{\pdif}[2]{\frac{\partial #1}{\partial #2}}
\newcommand{\ddif}{{\rm d}}
\newcommand{\Ketbra}[2]{\Ket{#1} \! \! \Bra{#2}}
\DeclareMathOperator{\Div}{div}
\DeclareMathOperator{\Grad}{grad}
\DeclareMathOperator{\Rot}{rot}
\renewcommand{\proofname}{証明}
\allowdisplaybreaks[4] 	 % 数式等のページ分割をさせる

% 定理環境
\newtcbtheorem{thm}{定理}{
coltitle = white,
colback = white,
colframe = black!50,
fonttitle = \bfseries,
breakable = true,
}{thm}
\newtcbtheorem[use counter from = thm]{dfn}{定義}{
coltitle = white,
colback = white,
colframe = black!50,
fonttitle = \bfseries,
breakable = true,
}{def}
\newtcbtheorem[use counter from = thm]{lem}{補題}{
coltitle = white,
colback = white,
colframe = black!50,
fonttitle = \bfseries,
breakable = true,
}{lem}
\newtcbtheorem[use counter from = thm]{prop}{命題}{
coltitle = white,
colback = white,
colframe = black!50,
fonttitle = \bfseries,
breakable = true,
}{prop}
\newtcbtheorem[use counter from = thm]{cor}{系}{
coltitle = white,
colback = white,
colframe = black!50,
fonttitle = \bfseries,
breakable = true,
}{cor}
\newtcbtheorem[use counter from = thm]{ass}{仮定}{
coltitle = white,
colback = white,
colframe = black!50,
fonttitle = \bfseries,
breakable = true,
}{ass}
\newtcbtheorem[use counter from = thm]{conj}{予想}{
coltitle = white,
colback = white,
colframe = black!50,
fonttitle = \bfseries,
breakable = true,
}{conj}

% https://marukunalufd0123.hatenablog.com/entry/2019/03/15/071717
\newcounter{problemNum}
\newtcolorbox{problem}[1][]{enhanced,
	breakable,
	boxrule=0.5mm,
	top=2pt,left=44pt,right=4pt,bottom=2pt,arc=0mm,
	colframe=blue!30!gray,
	boxrule=1pt,
	#1,
	underlay unbroken and first={
	\node[inner sep=1pt,blue!50!black,fill=blue!10!white]at ([xshift=22pt,yshift=-9pt]interior.north west) {\stepcounter{problemNum}\bfseries\gtfamily 問題\theproblemNum};},
	segmentation code={
	\draw[dashed] (segmentation.west)--(segmentation.east);
	\node[inner sep=1pt,blue!50!black,fill=blue!10!white] at ([xshift=22pt,yshift=-8pt]segmentation.south west) {\bfseries\gtfamily 解};},
	skin first is subskin of={enhancedfirst}{segmentation code={
	\draw[dashed] (segmentation.west)--(segmentation.east);
	\node[inner sep=1pt,blue!50!black,fill=blue!10!white] at ([xshift=22pt,yshift=-8pt]segmentation.south west) {\bfseries\gtfamily 解};}},
	before upper={\setlength{\parindent}{1zw}},
	before lower={\setlength{\parindent}{1zw}},
}

\DeclareMathOperator*{\Tr}{Tr}

\title{Schmidt分解}
\author{}

\begin{document}
\maketitle

\begin{thm}{Schmidt分解}{Schmidt分解}
	$N_A$準位系$A$と$N_B$準位系$B$の合成系の状態$\Ket{\Psi}_{AB}$は、
	$A$系の正規直交基底$\{\Ket{n}\}_{n=1}^{N_A}$と
	$B$系の正規直交基底$\{\Ket{u_n}\}_{n=1}^{N_B}$、
	正の整数$1 \leq r \leq \min (N_A, N_B)$と、
	$\sum_{n} p_n = 1$を満たす確率分布が存在して
	\begin{align}
		\Ket{\Psi}_{AB}
		= \sum_{n=1}^{r} \sqrt{p_n} \Ket{n}_A \Ket{u_n}_B
	\end{align}
	とできる。
\end{thm}
\begin{proof}
	一般性を失わずに$N_A \leq N_B$とできる。
	$\{\Ket{\psi_n}\}_{n=1}^{N_A}$、$\{\Ket{\phi_n}\}_{n=1}^{N_B}$を
	それぞれ$A$系、$B$系の正規直交基底とする。
	すれば
	\begin{align}
		\Ket{\Psi}_{AB}
		= \sum_{n, m} \alpha_{nm} \Ket{\psi_n}_A \Ket{\phi_m}_B
	\end{align}
	と展開できる。
	ここで
	\begin{align}
		\Ket{\chi_n} \coloneqq \sum_{m=1}^{N_B} \alpha_{nm} \Ket{\phi_m}
	\end{align}
	を用いると
	\begin{align}
		\Ket{\Psi}_{AB}
		= \sum_{n=1}^{N_A} \Ket{\psi_n}_A \Ket{\chi_n}_B
	\end{align}
	と書ける。
	また$X = (x_{ij}) \eqqcolon \Braket{\chi_i | \chi_j}$で定義される
	行列$X$を考えと
	\begin{align}
		\Braket{\eta | X | \eta}
		&= \sum_{i,j} \Braket{\eta | e_i}
		\Braket{e_i | X | e_j} \Braket{e_j | \eta} \\
		&= \sum_{i,j} \Braket{\eta | e_i} \Braket{\chi_i | \chi_j}
		\Braket{e_j | \eta} \\
		&= \Braket{\sum_i \Braket{e_i | \eta} \chi_i |
		\sum_j \Braket{e_j | \eta} \chi_j} \\
		&= \left\| \sum_i \Braket{e_i | \eta} \chi_i \right\|^2 \\
		&\geq 0
	\end{align}
	であるから、$X \geq 0$であることが分かる。
	$X$を対角化するユニタリ行列を$U = (u_{ij})$として
	\begin{gather}
		\Ket{n} \coloneqq \sum_{i=1}^{N_A} u^*_{in} \Ket{\psi_i} \\
		\Ket{u'_n} \coloneqq \sum_{i=1}^{N_A} u_{in} \Ket{\chi_i}
	\end{gather}
	を定めると、
	$\{\Ket{n}\}_{n=1}^{N_A}$は正規直交基底をなすことが分かる。
	また
	\begin{align}
		\Ket{\chi_i} = \sum_{n=1}^{N_A} u^*_{in} \Ket{u'_n}
	\end{align}
	である。

	以上から
	\begin{align}
		\Ket{\Psi}_{AB}
		&= \sum_{i=1}^{N_A} \Ket{\psi_i}_A \otimes
		\sum_{n=1}^{N_A} u^*_{in} \Ket{u'_n}_B \\
		&= \sum_{n=1}^{N_A} \Ket{n}_A \Ket{u'_n}
	\end{align}
	と書ける。
	この$\Ket{u'_n}$に注目すると
	\begin{align}
		\Braket{u'_n | u'_m}
		&= \sum_{i,j} u^*_{in} \Bra{\chi_i} u_{jm} \Ket{\chi_j} \\
		&= \sum_{i,j} u^*_{in} x_{ij} u_{jm} \\
		&= \left( U^\dagger X U \right)_{nm} \\
		&= p_n \delta_{nm}
	\end{align}
	となる。
	($\because$ $U$は$X$を対角化する。)
	また$X \geq 0$であったから$p_n \geq 0$である。
	順番を並び替えて、
	$1 \leq n \leq r$までは$p_n > 0$、
	$r + 1 \leq n \leq N_A$では$p_n = 0$となるようにし、
	$1 \leq n \leq r$について$\Ket{u_n} \coloneqq \Ket{u'_n} / \sqrt{p_n}$と
	新しく取り替えると結局
	\begin{align}
		\Ket{\Psi}_{AB} = \sum_{n=1}^{r} \sqrt{p_n} \Ket{n}_A \Ket{u_n}_B
	\end{align}
	と書き直せる。
	ここで規格化条件から
	\begin{align}
		\Braket{\Psi | \Psi}_{AB}
		= \sum_{n=1}^{r} p_n
		= 1
	\end{align}
	である。
	また$B$系の基底については、
	$\{u_n\}_{n=1}^{r}$に$N_B - r$個の基底を付け加えて新たな正規直交基底
	$\{u_n\}_{n=1}^{N_B}$を作ることが出来る。
\end{proof}

\end{document}