\documentclass[a4paper, 10pt]{jsarticle}
% 余白
\usepackage[top=20truemm, bottom=25truemm, left=22truemm, right=22truemm, driver=dvipdfm, truedimen, margin=2cm]{geometry}
% 数式
\usepackage{amsmath, amssymb, amsthm}
\usepackage{ascmac}
\usepackage{mathtools}
\usepackage{braket}
\mathtoolsset{showonlyrefs,showmanualtags} 	 % 相互参照した式のみに番号を振る
% 画像
\usepackage[dvipdfmx]{graphicx}
\usepackage[subrefformat=parens]{subcaption}
\captionsetup{compatibility=false}
% ハイパーリンク
\usepackage[dvipdfmx, bookmarksnumbered]{hyperref}
\usepackage{pxjahyper}
\hypersetup{colorlinks=true, linkcolor=black, citecolor=black, urlcolor=black}
% ページを跨ぐ枠
\usepackage{tcolorbox}
\tcbuselibrary{breakable, skins, theorems}

% コマンド定義
\def\vec#1{\mbox{\boldmath $#1$}}
\newcommand{\dif}[2]{\frac{{\rm d} #1}{{\rm d} #2}}
\newcommand{\pdif}[2]{\frac{\partial #1}{\partial #2}}
\newcommand{\ddif}{{\rm d}}
\newcommand{\Ketbra}[2]{\Ket{#1} \! \! \Bra{#2}}
\DeclareMathOperator{\Div}{div}
\DeclareMathOperator{\Grad}{grad}
\DeclareMathOperator{\Rot}{rot}
\renewcommand{\proofname}{証明}
\allowdisplaybreaks[4] 	 % 数式等のページ分割をさせる

% 定理環境
\newcounter{thetcbcounter}
\newtcbtheorem{thm}{定理}{
coltitle = white,
colback = white,
colframe = black!50,
fonttitle = \bfseries,
breakable = true,
}{thm}
\newtcbtheorem[use counter from = thm]{dfn}{定義}{
coltitle = white,
colback = white,
colframe = black!50,
fonttitle = \bfseries,
breakable = true,
}{def}
\newtcbtheorem[use counter from = thm]{lem}{補題}{
coltitle = white,
colback = white,
colframe = black!50,
fonttitle = \bfseries,
breakable = true,
}{lem}
\newtcbtheorem[use counter from = thm]{prop}{命題}{
coltitle = white,
colback = white,
colframe = black!50,
fonttitle = \bfseries,
breakable = true,
}{prop}
\newtcbtheorem[use counter from = thm]{cor}{系}{
coltitle = white,
colback = white,
colframe = black!50,
fonttitle = \bfseries,
breakable = true,
}{cor}
\newtcbtheorem[use counter from = thm]{ass}{仮定}{
coltitle = white,
colback = white,
colframe = black!50,
fonttitle = \bfseries,
breakable = true,
}{ass}
\newtcbtheorem[use counter from = thm]{conj}{予想}{
coltitle = white,
colback = white,
colframe = black!50,
fonttitle = \bfseries,
breakable = true,
}{conj}

\title{純粋状態と混合状態について}
\author{}

\begin{document}
\maketitle

\section{純粋状態}
\subsection{基準測定の$k$番目に対応する純粋状態}
まず一般の$N$準位系において、
基準測定の$k$番目が出る純粋状態、
つまり$p(l) = \delta_{kl}$の場合の密度演算子がどう表されるかを考える。
このとき$1 \leq a \leq N-1$である$\hat{\lambda}_a$について、
\begin{align}
	\Braket{\lambda_a}
	= \sum_{i=1}^{N} \lambda_a(i) p(i)
	= \lambda_a (k)
\end{align}
であるから
\begin{align}
	\hat{\rho}
	&= \frac{1}{N} \left[ \hat{I}
	+ \sum_{n=1}^{N^2-1} \Braket{\lambda_n} \hat{\lambda}_n \right] \\
	&= \frac{1}{N} \left[ \hat{I}
	+ \sum_{a=1}^{N-1} \lambda_a (k) \hat{\lambda}_a \right]
	+ \frac{1}{N} \sum_{n=N}^{N^2-1} \Braket{\lambda_n} \hat{\lambda}_n \\
	&= \Ketbra{k}{k}
	+ \frac{1}{N} \sum_{n=N}^{N^2-1} \Braket{\lambda_n} \hat{\lambda}_n
	\label{eq:dens_op_of_k}
\end{align}
と書ける。
ここで純粋状態である場合
$\Tr \left[ \hat{\rho}^2 \right] = 1$であることを用いると、
\begin{align}
	\Tr \left[ \hat{\rho}^2 \right]
	&= \Tr \left[ \Ketbra{k}{k}
	+ \frac{1}{N} \Ketbra{k}{k}
	\sum_{n=N}^{N^2-1} \Braket{\lambda_n} \hat{\lambda}_n
	+ \frac{1}{N} \sum_{n=N}^{N^2-1} \Braket{\lambda_n} \hat{\lambda}_n
	\Ketbra{k}{k}
	+ \frac{1}{N^2} \sum_{n=N}^{N^2-1} \Braket{\lambda_n} \hat{\lambda}_n
	\sum_{m=N}^{N^2-1} \Braket{\lambda_m} \hat{\lambda}_m \right] \\
	&= 1 + \frac{2}{N} \sum_{n=N}^{N^2-1} \Braket{\lambda_n}
	\Braket{k | \hat{\lambda}_n | k}
	+ \frac{1}{N} \sum_{n=N}^{N^2-1} \Braket{\lambda_n}^2 \\
	&= 1
\end{align}
つまり
\begin{align}
	2 \sum_{n=N}^{N^2-1} \Braket{\lambda_n} \Braket{k | \hat{\lambda}_n | k}
	+ \sum_{n=N}^{N^2-1} \Braket{\lambda_n}^2
	= 0
	\label{eq:condition}
\end{align}
でなければならないことが分かる。

ここで$N \leq n \leq N^2 - 1$について
$\Braket{k | \hat{\lambda}_n | k} = 0$であることを示す。
$1 \leq a \leq N-1$、$N \leq n \leq N^2-1$を満たす$a$、$n$について、
$\Tr \left[ \hat{\lambda}_a \hat{\lambda}_n \right] = 0$
でなければならないから、
\begin{align}
	\Tr \left[ \hat{\lambda}_a \hat{\lambda}_n \right]
	&= \sum_{k=1}^{N} \Braket{k | \hat{\lambda}_a \hat{\lambda}_n | k} \\
	&= \sum_{k=1}^{N} \sum_{l=1}^{N} \Braket{k | \hat{\lambda}_a | l}
	\Braket{l | \hat{\lambda}_n | k} \\
	&=  \sum_{k=1}^{N} \sum_{l=1}^{N} \lambda_a (k) \delta_{kl}
	\Braket{l | \hat{\lambda}_n | k} \\
	&= \sum_{k=1}^{N} \lambda_a (k) \Braket{k | \hat{\lambda}_n | k} \\
	&= 0
\end{align}
これが$\Braket{k | \hat{\lambda}_n | k} \neq 0$である場合に成り立つためには、
$N$次元内積空間の性質からある定数$c$、$1 \leq a' \leq N-1$が存在して
$\Braket{k | \hat{\lambda}_a | k} = c \lambda_{a'}(k)$となるか、
$\Braket{k | \hat{\lambda} | k} = c$とならなければならないが、
これは
\begin{gather}
	\Tr \left[ \hat{\lambda}_a \hat{\lambda}_n \right] = 0 \\
	\Tr \left[ \hat{\lambda}_n \right] = 0
\end{gather}
のいずれかに反する。
したがって$\Braket{k | \hat{\lambda}_n | k}$は0でなければならない。

以上のことから、式\eqref{eq:condition}が満たされるためには
\begin{align}
	\sum_{n=N}^{N^2 - 1} \Braket{\lambda_n}^2 = 0
\end{align}
すなわち、任意の$N \leq n \leq N^2 - 1$に対して
\begin{align}
	\Braket{\lambda_n} = 0
\end{align}
でなければならないことが分かる。
したがって、基準測定で$k$番目が出る純粋状態は式\eqref{eq:dens_op_of_k}より
\begin{align}
	\hat{\rho} = \Ketbra{k}{k}
\end{align}
と書ける。

\subsection{純粋状態における物理量の確率分布} \label{ss:prob}
純粋状態は密度演算子$\hat{\rho}$が、
ある単位ベクトル$\Ket{\psi}$を用いて
$\hat{\rho} = \Ketbra{\psi}{\psi}$と書ける状態である。
このときある物理量$A$が
\begin{gather}
	\hat{A} = \sum_{i=1}^{N} a_i \Ketbra{u_i}{u_i}
	= \sum_a a \hat{P}(a) \\
	\hat{P}(a) = \sum_{i a_i = a} \Ketbra{u_i}{u_i}
\end{gather}
とスペクトル分解されるとき、
測定値$a$が観測される確率は
\begin{align}
	\Pr \left[ A = a \right] = \Tr \left[ \hat{\rho} \hat{P}(a) \right]
	= \Braket{\psi | \hat{P}(a) | \psi}
\end{align}
と書ける。

\section{古典確率混合}
今までに見てきたように、
任意の単位ベクトル$\Ket{\psi}$に対応した$N$準位系の状態は、
基準測定で$k$番目の状態と$N$次ユニタリ行列に対応した物理操作によって準備できる。
しかし物理的に用意できる状態はそれだけではない。
これらの純粋状態が古典確率混合された状態も許されるはずである。

例えば確率$p_k$で$k$番目の状態を用意し、
それを何番目の状態かを教えずに誰か(ここでは仮にAlice)に渡す場合を考える。
このときAliceが節\ref{ss:prob}で出てきた物理量$A$を測定した場合に、
測定値が$a$である確率を考える。
もしもらった状態が$\hat{\rho}_k = \Ketbra{k}{k}$という状態であれば、
$a$を観測する確率は今まで見てきたように
$\Tr \left[ \hat{\rho} \hat{P}_a \right]$で与えられる。
しかしいま$\hat{\rho}_k$が貰える確率は$p_k$であるから、
結局$a$を観測する確率は
\begin{align}
	\Pr \left[ A = a \right]
	&= \sum_{k=1}^{N} p_k \Tr \left[ \hat{\rho}_k \hat{P}_a \right] \\
	&= \Tr \left[ \left( 
		\sum_{k=1}^{N} p_k \hat{\rho}_k
	 \right) \hat{P}_a \right]
\end{align}
となることが分かる。
これは
\begin{gather}
	\hat{\rho} \coloneqq \sum_{k=1}^{N} p_k \hat{\rho}_k
\end{gather}
という新しい密度演算子$\hat{\rho}$をもらったと考えるのと同等である。
この$\hat{\rho}$は
\begin{gather}
	\hat{\rho}^{\dagger} = \sum_{k=1}^{N} p_k \hat{\rho}_k^{\dagger}
	= \hat{\rho} \\
	\Tr \left[ \hat{\rho} \right]
	= \sum_{k=1}^{N} p_k \Tr \left[ \hat{\rho}_k \right]
	= \sum_{k=1}^{N} p_k = 1 \\
	\Braket{\psi | \hat{\rho} | \psi}
	= \sum_{k=1}^{N} p_k \Braket{\psi | \hat{\rho}_k | \psi}
	\geq 0
\end{gather}
を満たすから密度演算子である。

逆に今までの意味での混合状態
($\Tr \left[ \hat{\rho}^2 \right] < 1$の状態)を考えると、
$0 \leq p_i < 1$、$\sum_{i_1}^{N} p_i = 1$を満たす固有値$p_i$と、
それに対応する正規直交基底$\{ u_i \}_{i=1}^N$が存在して
\begin{gather}
	\hat{\rho} = \sum_{i=1}^N p_i \Ketbra{u_i}{u_i}
\end{gather}
と書け、
物理量$A$を測定したときに測定値$a$を得る確率は
\begin{align}
	\Pr \left[ A = a \right]
	&= \Tr \left[ \hat{\rho} \hat{P}_a \right] \\
	&= \Tr \left[ \left( \sum_{i=1}^{N} p_i \Ketbra{u_i}{u_i} \right)
	\hat{P}_a \right] \\
	&= \sum_{i=1}^N p_i \Tr \left[ \Ketbra{u_i}{u_i} \hat{P}_a \right]
\end{align}
であることが分かる。
これは確率$p_i$で状態$\Ketbra{u_i}{u_i}$を貰う場合と、
任意の物理量の測定値の出確率分布が一致するため、
物理的には同一の状態とみなせる(違いを見分けられない)。
これが混合状態の意味となる。

\begin{tcolorbox}[
enhanced,
colback = white,
boxrule = 0.5pt,
arc=2mm,
breakable
]
	\underline{補足}

	ちなみに$\hat{\rho}$の古典混合の仕方は一意には定まらない。
	例えば
	\begin{gather}
		\hat{\rho}
		= \frac{1}{3} \Ketbra{+}{+} + \frac{1}{3} \Ketbra{-}{-}
		+ \frac{1}{3} \Ketbra{u_+(\vec{e}_x)}{u_+(\vec{e}_x)} \\
		\hat{\rho}'
		= \frac{2}{3}  \Ketbra{u_+(\vec{e})}{u_+(\vec{e}_x)}
		+ \frac{1}{3} \Ketbra{u_-(\vec{e}_x)}{u_-(\vec{e}_x)}
	\end{gather}
	という2つの密度演算子は$\hat{\rho} = \hat{\rho}'$を満たす。
	このようなときには物理的に2つの状態を区別することはできない。
	そのことを見ておく。

	確率$p_i$で状態$\Ketbra{\phi_i}{\phi_i}$
	($1 \leq i \leq M$)である状態と、
	確率$q_i$で状態$\Ketbra{\psi_i}{\psi_i}$
	($1 \leq i \leq N$)である状態が
	密度演算子にすると等しい場合を考える。
	つまり
	\begin{align}
		\hat{\rho}
		= \sum_{i=1}^M p_i \Ketbra{\phi_i}{\phi_i}
		= \sum_{i=1}^N q_i \Ketbra{\psi_i}{\psi_i}
	\end{align}
	を満たす場合である。
	このとき物理量$A$の観測値$a$が得られる確率は
	\begin{align}
		\Pr \left[ A = a \right]
		&= \Tr \left[ \hat{\rho} \hat{P}_a \right] \\
		&= \sum_{i=1}^M p_i
		\Tr \left[ \Ketbra{\phi_i}{\phi_i} \hat{P}_a \right] \\
		&= \sum_{i=1}^N q_i
		\Tr \left[ \Ketbra{\psi_i}{\psi_i} \hat{P}_a \right]
	\end{align}
	となって測定値とその出現確率では見分けがつけられないことが分かる。
\end{tcolorbox}


\end{document}